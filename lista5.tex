\documentclass{article}

\usepackage[width=14cm, left=3cm, top=2cm]{geometry}

\usepackage[brazil]{babel}
\usepackage[T1]{fontenc}
\usepackage[utf8]{inputenc}

\usepackage{amsmath}
\usepackage{amssymb}

\usepackage{multicol}
\usepackage{multirow}

\usepackage{ifthen}
%\usepackage{minted}

\newcommand{\blank}{\rule[0pt]{5em}{.3pt}}
\newcommand{\nobody}{ninguém}
\newcommand{\preamble}[2]{\noindent%
    Fiz esse trabalho com a ajuda de {\bfseries #1}
    e consultei {\bfseries #2}.
    A versão final do trabalho foi feita
    por mim de forma independente.
    Respostas sem no mínimo 3 frases de justificativa
    não contam ponto.
    \par\noindent Assinatura: \blank\blank\bigskip}

\newcounter{exe-list}
\newenvironment{exe-list}
    {\begin{list}{\alph{exe-list}.}{\usecounter{exe-list}}}
    {\end{list}}

\newenvironment{exe}[2][Problema]
    {\newcommand{\opt}{(Opcional)}%
    \newcommand{\sketch}[1]{{\bfseries Rascunho:} ##1}%
    \medskip\par\noindent\ifthenelse{\equal{#1}{}}
        {\textbf{\large #2}}
        {\textbf{\large #1~#2}}%
    \medskip\par\noindent}
    {\medskip}

\newenvironment{dijkstra-table}{
    \begin{center}\begin{tabular}{|c|rcl|rcl|rcl|rcl|rcl|rcl|rcl|}
        \hline
        \multirow{2}{*}{\textbf{step}}
            & \multicolumn{3}{|c|}{\textbf{x}}
            & \multicolumn{3}{|c|}{\textbf{y}}
            & \multicolumn{3}{|c|}{\textbf{z}}
            & \multicolumn{3}{|c|}{\textbf{w}}
            & \multicolumn{3}{|c|}{\textbf{v}}
            & \multicolumn{3}{|c|}{\textbf{u}}
            & \multicolumn{3}{|c|}{\textbf{t}}
        \\
            & \textbf{V} & \textbf{D} & \textbf{p}
            & \textbf{V} & \textbf{D} & \textbf{p}
            & \textbf{V} & \textbf{D} & \textbf{p}
            & \textbf{V} & \textbf{D} & \textbf{p}
            & \textbf{V} & \textbf{D} & \textbf{p}
            & \textbf{V} & \textbf{D} & \textbf{p}
            & \textbf{V} & \textbf{D} & \textbf{p}
            \\\hline}
    {\end{tabular}\end{center}}

\newenvironment{distrib-table}{
    \begin{center}\begin{tabular}{|c||lr|c|lr|}\hline
        &\textbf{x}&\textbf{y}&\textbf{z}&\textbf{v}&\textbf{u}
        \\\hline\hline}
    {\end{tabular}\end{center}}

\title{Redes de Computadores - Lista 5}
\author{Daniel Kiyoshi Hashimoto Vouzella de Andrade - 119025937}
\date{Novembro 2022}

\begin{document}
\maketitle

%\preamble{\blank}{\blank}

\begin{exe}{0}
    Escolhendo um nó inicial \(i\):
    \begin{enumerate}
    \setcounter{enumi}{-1}
    \item Inicializa tabela com colunas: \par
        Nó \(v\),
        Visitado = \(F\),
        Custo = \((v == i) \:?\: 0 \::\: \infty\),
        Caminho = Nil
    \item Enquando existir nó \(v\) não visitado com menor custo:
        \begin{enumerate}
        \item Visitado \(v\) = \(T\)
        \item Para cada Nó \(u\) não visitado e vizinho de \(v\):
            \begin{enumerate}
                \item \(newcost :=\) Custo \(v\) \(+\) Dist \(v, u\)
                \item Se Custo \(u\) \(< newcost\):
                    \begin{itemize}
                        \item Custo \(u\) = \(newcost\)
                        \item Path \(u\) = Path \(v \cdot v\)
                    \end{itemize}
            \end{enumerate}
        \end{enumerate}
    \end{enumerate}
\end{exe}

\begin{exe}{3}
    \begin{dijkstra-table}
        $ -1 $
            &$ \bot $&$ 0 $&$ -1 $
            &$ \bot $&$ \infty $&
            &$ \bot $&$ \infty $&
            &$ \bot $&$ \infty $&
            &$ \bot $&$ \infty $&
            &$ \bot $&$ \infty $&
            &$ \bot $&$ \infty $&
            \\\hline
        $ 0 $
            &$ \top $&$ 0 $&$ -1 $
            &$ \bot $&$ 6 $& x
            &$ \bot $&$ 8 $& x
            &$ \bot $&$ 6 $& x
            &$ \bot $&$ 3 $& x
            &$ \bot $&$ \infty $&
            &$ \bot $&$ \infty $&
            \\\hline
        $ 1 $
            &$ \top $&$ 0 $&$ -1 $
            &$ \bot $&$ 6 $& x
            &$ \bot $&$ 8 $& x
            &$ \bot $&$ 6 $& x
            &$ \top $&$ 3 $& x
            &$ \bot $&$ 6 $& v
            &$ \bot $&$ 7 $& v
            \\\hline
        $ 2 $
            &$ \top $&$ 0 $&$ -1 $
            &$ \top $&$ 6 $& x
            &$ \bot $&$ 8 $& x
            &$ \bot $&$ 6 $& x
            &$ \top $&$ 3 $& x
            &$ \bot $&$ 6 $& v
            &$ \bot $&$ 7 $& v
            \\\hline
        $ 3 $
            &$ \top $&$ 0 $&$ -1 $
            &$ \top $&$ 6 $& x
            &$ \bot $&$ 8 $& x
            &$ \top $&$ 6 $& x
            &$ \top $&$ 3 $& x
            &$ \bot $&$ 6 $& v
            &$ \bot $&$ 7 $& v
            \\\hline
        $ 4 $
            &$ \top $&$ 0 $&$ -1 $
            &$ \top $&$ 6 $& x
            &$ \bot $&$ 8 $& x
            &$ \top $&$ 6 $& x
            &$ \top $&$ 3 $& x
            &$ \top $&$ 6 $& v
            &$ \bot $&$ 7 $& v
            \\\hline
        $ 5 $
            &$ \top $&$ 0 $&$ -1 $
            &$ \top $&$ 6 $& x
            &$ \bot $&$ 8 $& x
            &$ \top $&$ 6 $& x
            &$ \top $&$ 3 $& x
            &$ \top $&$ 6 $& v
            &$ \top $&$ 7 $& v
            \\\hline
        $ 6 $
            &$ \top $&$ 0 $&$ -1 $
            &$ \top $&$ 6 $& x
            &$ \top $&$ 8 $& x
            &$ \top $&$ 6 $& x
            &$ \top $&$ 3 $& x
            &$ \top $&$ 6 $& v
            &$ \top $&$ 7 $& v
            \\\hline
    \end{dijkstra-table}
\end{exe}

\begin{exe}{4}
    \begin{exe-list}
    \item Começando pelo \textbf{t} \par
    \begin{dijkstra-table}
        $ -1 $
            &$ \bot $&$ \infty $&
            &$ \bot $&$ \infty $&
            &$ \bot $&$ \infty $&
            &$ \bot $&$ \infty $&
            &$ \bot $&$ \infty $&
            &$ \bot $&$ \infty $&
            &$ \bot $&$ 0 $&$ -1 $
            \\\hline
        $ 0 $
            &$ \bot $&$ \infty $&
            &$ \bot $&$ 7 $& t
            &$ \bot $&$ \infty $&
            &$ \bot $&$ \infty $&
            &$ \bot $&$ 4 $& t
            &$ \bot $&$ 2 $& t
            &$ \top $&$ 0 $&$ -1 $
            \\\hline
        $ 1 $
            &$ \bot $&$ \infty $&
            &$ \bot $&$ 7 $& t
            &$ \bot $&$ \infty $&
            &$ \bot $&$ 5 $& u
            &$ \bot $&$ 4 $& t
            &$ \top $&$ 2 $& t
            &$ \top $&$ 0 $&$ -1 $
            \\\hline
        $ 2 $
            &$ \bot $&$ 7 $& v
            &$ \bot $&$ 7 $& t
            &$ \bot $&$ \infty $&
            &$ \bot $&$ 5 $& u
            &$ \top $&$ 4 $& t
            &$ \top $&$ 2 $& t
            &$ \top $&$ 0 $&$ -1 $
            \\\hline
        $ 3 $
            &$ \bot $&$ 7 $& v
            &$ \bot $&$ 7 $& t
            &$ \bot $&$ \infty $&
            &$ \top $&$ 5 $& u
            &$ \top $&$ 4 $& t
            &$ \top $&$ 2 $& t
            &$ \top $&$ 0 $&$ -1 $
            \\\hline
        $ 4 $
            &$ \top $&$ 7 $& v
            &$ \bot $&$ 7 $& t
            &$ \bot $&$ 15 $& x
            &$ \top $&$ 5 $& u
            &$ \top $&$ 4 $& t
            &$ \top $&$ 2 $& t
            &$ \top $&$ 0 $&$ -1 $
            \\\hline
        $ 5 $
            &$ \top $&$ 7 $& v
            &$ \top $&$ 7 $& t
            &$ \bot $&$ 15 $& x
            &$ \top $&$ 5 $& u
            &$ \top $&$ 4 $& t
            &$ \top $&$ 2 $& t
            &$ \top $&$ 0 $&$ -1 $
            \\\hline
        $ 6 $
            &$ \top $&$ 7 $& v
            &$ \top $&$ 7 $& t
            &$ \top $&$ 15 $& x
            &$ \top $&$ 5 $& u
            &$ \top $&$ 4 $& t
            &$ \top $&$ 2 $& t
            &$ \top $&$ 0 $&$ -1 $
            \\\hline
    \end{dijkstra-table}
    \item Começando pelo \textbf{u} \par
    \begin{dijkstra-table}
        $ -1 $
            &$ \bot $&$ \infty $&
            &$ \bot $&$ \infty $&
            &$ \bot $&$ \infty $&
            &$ \bot $&$ \infty $&
            &$ \bot $&$ \infty $&
            &$ \bot $&$ 0 $&$ -1 $
            &$ \bot $&$ \infty $&
            \\\hline
        $ 0 $
            &$ \bot $&$ \infty $&
            &$ \bot $&$ \infty $&
            &$ \bot $&$ \infty $&
            &$ \bot $&$ 3 $& u
            &$ \bot $&$ 3 $& u
            &$ \top $&$ 0 $&$ -1 $
            &$ \bot $&$ 2 $& u
            \\\hline
        $ 1 $
            &$ \bot $&$ \infty $&
            &$ \bot $&$ 9 $& t
            &$ \bot $&$ \infty $&
            &$ \bot $&$ 3 $& u
            &$ \bot $&$ 3 $& u
            &$ \top $&$ 0 $&$ -1 $
            &$ \top $&$ 2 $& u
            \\\hline
        $ 2 $
            &$ \bot $&$ 9 $& w
            &$ \bot $&$ 9 $& t
            &$ \bot $&$ \infty $&
            &$ \top $&$ 3 $& u
            &$ \bot $&$ 3 $& u
            &$ \top $&$ 0 $&$ -1 $
            &$ \top $&$ 2 $& u
            \\\hline
        $ 3 $
            &$ \bot $&$ 6 $& v
            &$ \bot $&$ 9 $& t
            &$ \bot $&$ \infty $&
            &$ \top $&$ 3 $& u
            &$ \top $&$ 3 $& u
            &$ \top $&$ 0 $&$ -1 $
            &$ \top $&$ 2 $& u
            \\\hline
        $ 4 $
            &$ \top $&$ 6 $& v
            &$ \bot $&$ 9 $& t
            &$ \bot $&$ 14 $& x
            &$ \top $&$ 3 $& u
            &$ \top $&$ 3 $& u
            &$ \top $&$ 0 $&$ -1 $
            &$ \top $&$ 2 $& u
            \\\hline
        $ 5 $
            &$ \top $&$ 6 $& v
            &$ \top $&$ 9 $& t
            &$ \bot $&$ 14 $& x
            &$ \top $&$ 3 $& u
            &$ \top $&$ 3 $& u
            &$ \top $&$ 0 $&$ -1 $
            &$ \top $&$ 2 $& u
            \\\hline
        $ 6 $
            &$ \top $&$ 6 $& v
            &$ \top $&$ 9 $& t
            &$ \top $&$ 14 $& x
            &$ \top $&$ 3 $& u
            &$ \top $&$ 3 $& u
            &$ \top $&$ 0 $&$ -1 $
            &$ \top $&$ 2 $& u
            \\\hline
    \end{dijkstra-table}
    \item Começando pelo \textbf{v} \par
    \begin{dijkstra-table}
        $ -1 $
            &$ \bot $&$ \infty $&
            &$ \bot $&$ \infty $&
            &$ \bot $&$ \infty $&
            &$ \bot $&$ \infty $&
            &$ \bot $&$ 0 $&$ -1 $
            &$ \bot $&$ \infty $&
            &$ \bot $&$ \infty $&
            \\\hline
        $ 0 $
            &$ \bot $&$ 3 $& v
            &$ \bot $&$ 8 $& v
            &$ \bot $&$ \infty $&
            &$ \bot $&$ 4 $& v
            &$ \top $&$ 0 $&$ -1 $
            &$ \bot $&$ 3 $& v
            &$ \bot $&$ 4 $& v
            \\\hline
        $ 1 $
            &$ \top $&$ 3 $& v
            &$ \bot $&$ 8 $& v
            &$ \bot $&$ 11 $& x
            &$ \bot $&$ 4 $& v
            &$ \top $&$ 0 $&$ -1 $
            &$ \bot $&$ 3 $& v
            &$ \bot $&$ 4 $& v
            \\\hline
        $ 2 $
            &$ \top $&$ 3 $& v
            &$ \bot $&$ 8 $& v
            &$ \bot $&$ 11 $& x
            &$ \bot $&$ 4 $& v
            &$ \top $&$ 0 $&$ -1 $
            &$ \top $&$ 3 $& v
            &$ \bot $&$ 4 $& v
            \\\hline
        $ 3 $
            &$ \top $&$ 3 $& v
            &$ \bot $&$ 8 $& v
            &$ \bot $&$ 11 $& x
            &$ \top $&$ 4 $& v
            &$ \top $&$ 0 $&$ -1 $
            &$ \top $&$ 3 $& v
            &$ \bot $&$ 4 $& v
            \\\hline
        $ 4 $
            &$ \top $&$ 3 $& v
            &$ \bot $&$ 8 $& v
            &$ \bot $&$ 11 $& x
            &$ \top $&$ 4 $& v
            &$ \top $&$ 0 $&$ -1 $
            &$ \top $&$ 3 $& v
            &$ \top $&$ 4 $& v
            \\\hline
        $ 5 $
            &$ \top $&$ 3 $& v
            &$ \top $&$ 8 $& v
            &$ \bot $&$ 11 $& x
            &$ \top $&$ 4 $& v
            &$ \top $&$ 0 $&$ -1 $
            &$ \top $&$ 3 $& v
            &$ \top $&$ 4 $& v
            \\\hline
        $ 6 $
            &$ \top $&$ 3 $& v
            &$ \top $&$ 8 $& v
            &$ \top $&$ 11 $& x
            &$ \top $&$ 4 $& v
            &$ \top $&$ 0 $&$ -1 $
            &$ \top $&$ 3 $& v
            &$ \top $&$ 4 $& v
            \\\hline
    \end{dijkstra-table}
    \item Começando pelo \textbf{w} \par
    \begin{dijkstra-table}
        $ -1 $
            &$ \bot $&$ \infty $&
            &$ \bot $&$ \infty $&
            &$ \bot $&$ \infty $&
            &$ \bot $&$ 0 $&$ -1 $
            &$ \bot $&$ \infty $&
            &$ \bot $&$ \infty $&
            &$ \bot $&$ \infty $&
            \\\hline
        $ 0 $
            &$ \bot $&$ 6 $& w
            &$ \bot $&$ \infty $&
            &$ \bot $&$ \infty $&
            &$ \top $&$ 0 $&$ -1 $
            &$ \bot $&$ 4 $& w
            &$ \bot $&$ 3 $& w
            &$ \bot $&$ \infty $&
            \\\hline
        $ 1 $
            &$ \bot $&$ 6 $& w
            &$ \bot $&$ \infty $&
            &$ \bot $&$ \infty $&
            &$ \top $&$ 0 $&$ -1 $
            &$ \bot $&$ 4 $& w
            &$ \top $&$ 3 $& w
            &$ \bot $&$ 5 $& u
            \\\hline
        $ 2 $
            &$ \bot $&$ 6 $& w
            &$ \bot $&$ 12 $& v
            &$ \bot $&$ \infty $&
            &$ \top $&$ 0 $&$ -1 $
            &$ \top $&$ 4 $& w
            &$ \top $&$ 3 $& w
            &$ \bot $&$ 5 $& u
            \\\hline
        $ 3 $
            &$ \bot $&$ 6 $& w
            &$ \bot $&$ 12 $& v
            &$ \bot $&$ \infty $&
            &$ \top $&$ 0 $&$ -1 $
            &$ \top $&$ 4 $& w
            &$ \top $&$ 3 $& w
            &$ \top $&$ 5 $& u
            \\\hline
        $ 4 $
            &$ \top $&$ 6 $& w
            &$ \bot $&$ 12 $& v
            &$ \bot $&$ 14 $& x
            &$ \top $&$ 0 $&$ -1 $
            &$ \top $&$ 4 $& w
            &$ \top $&$ 3 $& w
            &$ \top $&$ 5 $& u
            \\\hline
        $ 5 $
            &$ \top $&$ 6 $& w
            &$ \top $&$ 12 $& v
            &$ \bot $&$ 14 $& x
            &$ \top $&$ 0 $&$ -1 $
            &$ \top $&$ 4 $& w
            &$ \top $&$ 3 $& w
            &$ \top $&$ 5 $& u
            \\\hline
        $ 6 $
            &$ \top $&$ 6 $& w
            &$ \top $&$ 12 $& v
            &$ \top $&$ 14 $& x
            &$ \top $&$ 0 $&$ -1 $
            &$ \top $&$ 4 $& w
            &$ \top $&$ 3 $& w
            &$ \top $&$ 5 $& u
            \\\hline
    \end{dijkstra-table}
    \item Começando pelo \textbf{y} \par
    \begin{dijkstra-table}
        $ -1 $
            &$ \bot $&$ \infty $&
            &$ \bot $&$ 0 $&$ -1 $
            &$ \bot $&$ \infty $&
            &$ \bot $&$ \infty $&
            &$ \bot $&$ \infty $&
            &$ \bot $&$ \infty $&
            &$ \bot $&$ \infty $&
            \\\hline
        $ 0 $
            &$ \bot $&$ 6 $& y
            &$ \top $&$ 0 $&$ -1 $
            &$ \bot $&$ 12 $& z
            &$ \bot $&$ \infty $&
            &$ \bot $&$ 8 $& y
            &$ \bot $&$ \infty $&
            &$ \bot $&$ 7 $& y
            \\\hline
        $ 1 $
            &$ \top $&$ 6 $& y
            &$ \top $&$ 0 $&$ -1 $
            &$ \bot $&$ 12 $& z
            &$ \bot $&$ 12 $& x
            &$ \bot $&$ 8 $& y
            &$ \bot $&$ \infty $&
            &$ \bot $&$ 7 $& y
            \\\hline
        $ 2 $
            &$ \top $&$ 6 $& y
            &$ \top $&$ 0 $&$ -1 $
            &$ \bot $&$ 12 $& z
            &$ \bot $&$ 12 $& x
            &$ \top $&$ 8 $& y
            &$ \bot $&$ 11 $& v
            &$ \bot $&$ 7 $& y
            \\\hline
        $ 3 $
            &$ \top $&$ 6 $& y
            &$ \top $&$ 0 $&$ -1 $
            &$ \bot $&$ 12 $& z
            &$ \bot $&$ 12 $& x
            &$ \top $&$ 8 $& y
            &$ \bot $&$ 9 $& t
            &$ \top $&$ 7 $& y
            \\\hline
        $ 4 $
            &$ \top $&$ 6 $& y
            &$ \top $&$ 0 $&$ -1 $
            &$ \bot $&$ 12 $& z
            &$ \bot $&$ 11 $& u
            &$ \top $&$ 8 $& y
            &$ \top $&$ 9 $& t
            &$ \top $&$ 7 $& y
            \\\hline
        $ 5 $
            &$ \top $&$ 6 $& y
            &$ \top $&$ 0 $&$ -1 $
            &$ \bot $&$ 12 $& z
            &$ \top $&$ 11 $& u
            &$ \top $&$ 8 $& y
            &$ \top $&$ 9 $& t
            &$ \top $&$ 7 $& y
            \\\hline
        $ 6 $
            &$ \top $&$ 6 $& y
            &$ \top $&$ 0 $&$ -1 $
            &$ \top $&$ 12 $& z
            &$ \top $&$ 11 $& u
            &$ \top $&$ 8 $& y
            &$ \top $&$ 9 $& t
            &$ \top $&$ 7 $& y
            \\\hline
    \end{dijkstra-table}
    \item Começando pelo \textbf{z} \par
    \begin{dijkstra-table}
        $ -1 $
            &$ \bot $&$ \infty $&
            &$ \bot $&$ \infty $&
            &$ \bot $&$ 0 $&$ -1 $
            &$ \bot $&$ \infty $&
            &$ \bot $&$ \infty $&
            &$ \bot $&$ \infty $&
            &$ \bot $&$ \infty $&
            \\\hline
        $ 0 $
            &$ \bot $&$ 8 $& z
            &$ \bot $&$ 12 $& z
            &$ \top $&$ 0 $&$ -1 $
            &$ \bot $&$ \infty $&
            &$ \bot $&$ \infty $&
            &$ \bot $&$ \infty $&
            &$ \bot $&$ \infty $&
            \\\hline
        $ 1 $
            &$ \top $&$ 8 $& z
            &$ \bot $&$ 12 $& z
            &$ \top $&$ 0 $&$ -1 $
            &$ \bot $&$ 14 $& x
            &$ \bot $&$ 11 $& x
            &$ \bot $&$ \infty $&
            &$ \bot $&$ \infty $&
            \\\hline
        $ 2 $
            &$ \top $&$ 8 $& z
            &$ \bot $&$ 12 $& z
            &$ \top $&$ 0 $&$ -1 $
            &$ \bot $&$ 14 $& x
            &$ \top $&$ 11 $& x
            &$ \bot $&$ 14 $& v
            &$ \bot $&$ 15 $& v
            \\\hline
        $ 3 $
            &$ \top $&$ 8 $& z
            &$ \top $&$ 12 $& z
            &$ \top $&$ 0 $&$ -1 $
            &$ \bot $&$ 14 $& x
            &$ \top $&$ 11 $& x
            &$ \bot $&$ 14 $& v
            &$ \bot $&$ 15 $& v
            \\\hline
        $ 4 $
            &$ \top $&$ 8 $& z
            &$ \top $&$ 12 $& z
            &$ \top $&$ 0 $&$ -1 $
            &$ \top $&$ 14 $& x
            &$ \top $&$ 11 $& x
            &$ \bot $&$ 14 $& v
            &$ \bot $&$ 15 $& v
            \\\hline
        $ 5 $
            &$ \top $&$ 8 $& z
            &$ \top $&$ 12 $& z
            &$ \top $&$ 0 $&$ -1 $
            &$ \top $&$ 14 $& x
            &$ \top $&$ 11 $& x
            &$ \top $&$ 14 $& v
            &$ \bot $&$ 15 $& v
            \\\hline
        $ 6 $
            &$ \top $&$ 8 $& z
            &$ \top $&$ 12 $& z
            &$ \top $&$ 0 $&$ -1 $
            &$ \top $&$ 14 $& x
            &$ \top $&$ 11 $& x
            &$ \top $&$ 14 $& v
            &$ \top $&$ 15 $& v
            \\\hline
    \end{dijkstra-table}
    \end{exe-list}
\end{exe}

\begin{exe}{5}
    \begin{itemize}
        \item Passo 0:
            \begin{distrib-table}
                \textbf{x}&$ \infty $&$ \infty $&$ \infty $&$ \infty $&$ \infty $\\
                \textbf{y}&$ \infty $&$ \infty $&$ \infty $&$ \infty $&$ \infty $\\\hline
                \textbf{z}&$ 2 $&$ \infty $&$ 0 $&$ 6 $&$ \infty $\\\hline
                \textbf{v}&$ \infty $&$ \infty $&$ \infty $&$ \infty $&$ \infty $\\
                \textbf{u}&$ \infty $&$ \infty $&$ \infty $&$ \infty $&$ \infty $\\\hline
            \end{distrib-table}
        \item Passo 1:
            \begin{distrib-table}
                \textbf{x}&$ 0 $&$ 3 $&$ 2 $&$ 3 $&$ \infty $\\
                \textbf{y}&$ \infty $&$ \infty $&$ \infty $&$ \infty $&$ \infty $\\\hline
                \textbf{z}&$ 2 $&$ 5 $&$ 0 $&$ 5 $&$ 7 $\\\hline
                \textbf{v}&$ 3 $&$ \infty $&$ 6 $&$ 0 $&$ 1 $\\
                \textbf{u}&$ \infty $&$ \infty $&$ \infty $&$ \infty $&$ \infty $\\\hline
            \end{distrib-table}
        \item Passo 2:
            \begin{distrib-table}
                \textbf{x}&$ 0 $&$ 3 $&$ 2 $&$ 3 $&$ 4 $\\
                \textbf{y}&$ 3 $&$ 0 $&$ \infty $&$ \infty $&$ 2 $\\\hline
                \textbf{z}&$ 2 $&$ 5 $&$ 0 $&$ 5 $&$ 6 $\\\hline
                \textbf{v}&$ 3 $&$ 3 $&$ 5 $&$ 0 $&$ 1 $\\
                \textbf{u}&$ \infty $&$ 2 $&$ \infty $&$ 1 $&$ 0 $\\\hline
            \end{distrib-table}
        \item Passo 3:
            \begin{distrib-table}
                \textbf{x}&$ 0 $&$ 3 $&$ 2 $&$ 3 $&$ 4 $\\
                \textbf{y}&$ 3 $&$ 0 $&$ 5 $&$ 3 $&$ 2 $\\\hline
                \textbf{z}&$ 2 $&$ 5 $&$ 0 $&$ 5 $&$ 6 $\\\hline
                \textbf{v}&$ 3 $&$ 3 $&$ 5 $&$ 0 $&$ 1 $\\
                \textbf{u}&$ 4 $&$ 2 $&$ 7 $&$ 1 $&$ 0 $\\\hline
            \end{distrib-table}
        \item Passo 4:
            \begin{distrib-table}
                \textbf{x}&$ 0 $&$ 3 $&$ 2 $&$ 3 $&$ 4 $\\
                \textbf{y}&$ 3 $&$ 0 $&$ 5 $&$ 3 $&$ 2 $\\\hline
                \textbf{z}&$ 2 $&$ 5 $&$ 0 $&$ 5 $&$ 6 $\\\hline
                \textbf{v}&$ 3 $&$ 3 $&$ 5 $&$ 0 $&$ 1 $\\
                \textbf{u}&$ 4 $&$ 2 $&$ 6 $&$ 1 $&$ 0 $\\\hline
            \end{distrib-table}
        \item Passo 5:
            \begin{distrib-table}
                \textbf{x}&$ 0 $&$ 3 $&$ 2 $&$ 3 $&$ 4 $\\
                \textbf{y}&$ 3 $&$ 0 $&$ 5 $&$ 3 $&$ 2 $\\\hline
                \textbf{z}&$ 2 $&$ 5 $&$ 0 $&$ 5 $&$ 6 $\\\hline
                \textbf{v}&$ 3 $&$ 3 $&$ 5 $&$ 0 $&$ 1 $\\
                \textbf{u}&$ 4 $&$ 2 $&$ 6 $&$ 1 $&$ 0 $\\\hline
            \end{distrib-table}
    \end{itemize}
\end{exe}

\begin{exe}{6}
    Suponha que o grafo que representa a rede tenha
    que a maior distância entre quaisquer dois nós seja \(d\),
    desconsiderando os pesos das arestas,
    ou seja, liste todos os menores caminhos entre dois nós,
    \(d\) é o maior número dessa lista.
    Seja \(a\) um nó exemplo desse grafo.
    \par
    Sabemos que na iteração \(0\),
    o nó \(a\) sabe somente das arestas que levam até seus vizinhos e
    ninguém disse nada para ele.
    Na iteração \(1\), os vizinhos de \(a\) ``contam'' para ele
    como é o grafo na vizinhança deles,
    pois no final da iteração \(0\) eles mandam uma mensagem
    com essa descrição.
    Dessa forma, agora, \(a\) conhece ``todas''
    as arestas que ele consegue alcançar em \(2\) passos.
    Na iteração \(2\), ``chegam''
    as mensagens dos \(2\)-vizinhos de \(a\)
    (os nós a exatamente \(2\) arestas de distância),
    pois como estão a \(2\) de distância as mensagems demoram
    \(2\) iterações para chegar.
    Então ele conhece ``todas'' as arestas do grafo
    que ele chega em \(3\) passos.
    Assim, na iteração \(i\),
    \(a\) vai receber a mensagem dos \(i\)-vizinhos e
    vai conhecer ``todas'' as arestas
    que consegue alcançar em \(i+1\) passos.
    \par
    Note que não é exatamente assim que as coisas acontecem,
    por isso as aspas em vários lugares.
    Na verdade, no algorítmo e implementação,
    são encaminhadas informações
    mais abstradas e resumidas sobre a topologia do grafo,
    as distâncias (com pesos) entre cada nó,
    sendo atualizada a cada iteração.
    Entretanto essa analogia continua funcionando.
    \par
    No pior caso, existem pelo menos dois nós \(z_0, z_1\)
    com uma distância \(d\) de \(a\) para eles e
    existe uma aresta entre \(z_0\) e \(z_1\).
    \(a\) só vai conseguir alcançar o nó \(z_0\)
    na iteração \(d-1\),
    pois vai conhecer todos os caminhos de até \(d\) passos,
    portante vai conseguir alcançar/enviar uma mensagem para \(z_0\).
    Entretanto, não podemos garantir que esse é o melhor caminho,
    pois ainda não conseguimos perceber
    a aresta entre \(z_0\) e \(z_1\).
    Dessa forma, na próxima iteração \((d)\),
    \(a\) vai ``receber'' informação sobre essa aresta faltante
    e vai poder determinar o melhor caminho até \(z_0\)
    (ou qualquer outro nó).
    \par
    Mas note que as tabelas continuam dessincronizadas.
    Em uma condição mais realista
    (onde não se envia uma vizinhança inteira do grafo toda iteração),
    na iteração \(d\),
    \(a\) recebeu as informações de \(z_0\)
    referêntes a iteração \(0\),
    então na iteração \(d+1\), ele receberá informações
    referêntes a iteração \(1\),
    e assim por diante.
    De forma que na iteração \(2d\),
    todas as tabelas estejam sincronizadas.
\end{exe}

\begin{exe}{7}
    \begin{exe-list}
    \item \hspace{1ex}\par
        \begin{center}\begin{tabular}{|c|rcl|}\hline
            & \textbf{w} & \textbf{y} & \textbf{u} \\\hline
            \textbf{x} & 2 & 5 & 7 \\\hline
        \end{tabular}\end{center}
    \item Basta uma mudança em \(c(x, w)\) ou que faça:
        \[
            c(x, w) + 5 > c(x, y) + 6
            \quad\Longrightarrow\quad
            c(x, y) < c(x, w) - 1
        \]
        Note que alterar o \(c(x, w)\) vai melhorar ou piorar
        o caminho, então vai informar os vizinhos.
        Por exemplo: \(c(x, w) \leftarrow 7\).
    \item Basta uma mudança em \(c(x, y)\) que mantenha:
        \[
            c(x, w) + 5 \le c(x, y) + 6
            \quad\Longrightarrow\quad
            c(x, y) \ge c(x, w) - 1
        \]
        Note que alterar o \(c(x, w)\) vai melhorar ou piorar
        o caminho, então vai informar os vizinhos.
        Por exemplo: \(c(x, y) \leftarrow 6\)
    \end{exe-list}
\end{exe}

\begin{exe}{9}
    Diminuir o peso de um link entre nós \(u\) e \(v\)
    não causa o problema de \emph{count-to-infinity},
    pois isso vai diminuir \(D_u(v)\) e vai ser propagado.
    Durante a propagação no nó \(a\), temos dois casos:
    \(D_a(v)\) vai ser substituído por um valor menor
    e causar outra propagação;
    ou vai ser ignorado e não vai ser propagado.
    Um nó que já propagou, ao receber uma propagação,
    só vai alterar a tabela se receber ``uma proposta melhor''
    do que a atual.

    Uma outra explicação é que no problema de \emph{count-to-infinity}
    um nó \(v\) perde um caminho bom para o alvo \(t\),
    mas acredita que existe outro caminho que passa \(u\),
    então ele atualiza a sua tabela para passar por \(u\).
    Porém, o caminho de \(u\) para \(t\) passava por \(v\),
    então um nó acha que o outro sabe chegar no \(t\),
    mas na verdade nenhum dos dois sabe.
    Diminuir o peso de um link não faz um nó perder um caminho bom
    para o alvo, só encontra um melhor.

    Conectar dois nós que não estavam previamente conectados
    tem um efeito semelhante a chegar uma informação de que
    um vizinho conhece um caminho para o outro nó.
    Se esse caminho for melhor,
    vai acontecer algo análogo à diminuir o peso de um link.
    Se o caminho for pior,
    o nó vai ignorá-lo, pois já conhece um caminho melhor.

    Uma outra forma de pensar é dizer que o link sempre existiu,
    mas foi reduzido de \(\infty\) para uma quantidade ``finita''.
\end{exe}

\begin{exe}{9 - Continuação}
    O site mostra um conjunto de regras que tenta resolver
    o problema de \emph{count-to-infinity}.
    A ideia é contar uma ``mentira branca'' para os nós
    que fazem parte do caminho até o nó \(t\) sobre
    saber como chegar até \(t\);
    por exemplo, se o nó \(v\) repassa o pacote para \(u\)
    para alcançar o \(t\), \(v\) diz apenas para \(u\)
    que não sabe chegar em \(t\).
    Infelizmente essas regras só resolvem o problema
    em topoligias mais simples.
\end{exe}

\begin{exe}[Desafio]{}
    O objetivo da prova é mostrar que para um grafo \(G\)
    o algorítmo encontra corretamente
    o melhor caminho (mais curto/menor soma de pesos)
    que sai do nó inicial \(s\) e chega a outro nó \(t\).

    Durante a execução do algorítmo temos os conjuntos:
    \(N_i \subseteq V(G)\) o conjunto de nós visitados; e
    \(F_i \subseteq V(G)\) a fronteira,
    o conjunto de nós que podem ser
    alcançados a partir de um nó de \(N_i\) usando apenas uma aresta
    e não estão em \(N_i\).
    Nesses casos \(i\) é apenas um contador da iteração atual.

    Tentamos mostrar por indução que a seguinte propriedade se mantém:
    \emph{Em todos as iterações do algorítmo,
    o caminho escolhido entre \(s\) e outro nó de \(N_i\),
    tem um tamanho igual ou melhor ao tamanho do caminho ótimo.}
    \begin{itemize}
        \item \textbf{Caso base:} \(N_0 = \{ s \}\) \par
            A única escolha possível de ``outro nó'' é \(s\),
            o tamanho do caminho ótimo de \(s\) até \(s\) é \(0\),
            assim como dito pelo algorítmo.
        \item \textbf{Passo indutivo:} \(i\) \par
            Sabemos que todas as distâncias até os nós de \(N_i\)
            são ótimas.
            Estamos querendo adicionar um nó \(v \in F_i\) ao \(N_i\),
            de forma que \(N_{i+1} = N_i \cup \{v\}\),
            e o caminho escolhido até \(v\) passa por \(u \in N_i\)
            imediatamente antes de chegar em \(v\).

            Vamos assumir que no melhor caminho até \(v\),
            o último nó de \(N_i\) é \(x\) e
            o nó seguinte é \(y \in F_i\),
            e então podem ter \(0\) ou mais nós entre \(y\) e \(v\).
            Note que é possível que
            \(x\) seja \(s\) e/ou \(y\) seja \(v\).

            Sendo \(D_s(n)\) a distância de \(s\) até \(n \in V(G)\)
            escolhida pelo algorítmo,
            \(D^*_s(n)\) a distância ótima de \(s\) até \(n \in V(G)\)
            e \(C(n, m)\) é o custo da aresta entre \(n, m \in V(G)\),
            temos as seguintes afirmações:
            \begin{align}
                \label{sxy-le-opv}
                D_s(x) + C(x, y) \le D^*_s(v) \\
                \label{sy-le-sxy}
                D_s(y) \le D_s(x) +  C(x, y) \\
                \label{sv-le-sy}
                D_s(v) \le D_s(y)
            \end{align}
            (\ref{sxy-le-opv})
            é obtida através da hipotese de indução e
            da definição de caminho ótimo.
            (\ref{sy-le-sxy}) vêm da hipótese inicial sobre \(G\)
            que não possui arestas negativas e que \(y \in F_i\).
            Sabemos que (\ref{sv-le-sy}), pois \(y\) não foi escolhido
            pelo algorítmo até o presente momento,
            mas \(v\) foi escolhido agora.
            Podemos juntar as expressões da seguinte forma:
            \begin{align*}
                D_s(y) &\le D^*_s(v)
                    &&\text{Usando \ref{sy-le-sxy} e \ref{sxy-le-opv}}\\
                D_s(v) &\le D^*_s(v)
                    &&\text{Usando \ref{sv-le-sy}}\\
            \end{align*}
            Como, por definição, \(D_s(v) < D^*_s(v)\) é impossível,
            concluímos que \(D_s(v) = D^*_s(v)\).
            E a hipótese de indução continua valendo para
            a iteração \(i+1\).

            \emph{Note que foi usado uma expressão a menos,
            comparando com o vídeo.
            A expressão extra apenas adicionava artificialmente
            a ``conveniência'' de uma prova por contradição.}
    \end{itemize}
    Assim, sabemos que no final do algorítmo
    todos os caminhos escolhidos tem tamanho igual
    ao caminho ótimo.
\end{exe}

\end{document}
