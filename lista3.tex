\documentclass{article}

\usepackage[width=14cm, left=3cm, top=2cm]{geometry}

\usepackage[brazil]{babel}
\usepackage[T1]{fontenc}
\usepackage[utf8]{inputenc}

\usepackage{amsmath}
\usepackage{amssymb}

\usepackage{multicol}

\usepackage{ifthen}
%\usepackage{minted}

\newcommand{\blank}{\rule[0pt]{5em}{.3pt}}
\newcommand{\nobody}{ninguém}
\newcommand{\preamble}[2]{\noindent%
    Fiz esse trabalho com a ajuda de {\bfseries #1}
    e consultei {\bfseries #2}.
    A versão final do trabalho foi feita
    por mim de forma independente.
    Respostas sem no mínimo 3 frases de justificativa
    não contam ponto.
    \par\noindent Assinatura: \blank\blank\bigskip}

\newcounter{exe-list}
\newenvironment{exe-list}
    {\begin{list}{\alph{exe-list}.}{\usecounter{exe-list}}}
    {\end{list}}

\newenvironment{exe}[2][Problema]
    {\newcommand{\opt}{(Opcional)}%
    \newcommand{\sketch}[1]{{\bfseries Rascunho:} ##1}%
    \medskip\par\noindent\ifthenelse{\equal{#1}{}}
        {\textbf{\large #2}}
        {\textbf{\large #1~#2}}%
    \medskip\par\noindent}
    {\medskip}

\newcommand{\tab}[1]{\phantom{\hspace{#1}}}

\DeclareMathOperator{\ms}{ms}

\title{Redes de Computadores - Lista X}
\author{Daniel Kiyoshi Hashimoto Vouzella de Andrade - 119025937}
\date{Outubro 2022}

\begin{document}
\maketitle

%\preamble{\blank}{\blank}

\begin{exe}{6}
    Uma maneira de chegar ao deadlock é:
    \begin{enumerate}
        \item Transmissor: envia pacote com seq=0
        \item Receptor: recebe ok, envia ACK
        \item Transmissor: recebe corrompido,
            reenvia pacote com seq=0
    \end{enumerate}
    Agora estamos em um deadlock ``vivo'',
    onde o Receptor sempre vai mandar NAK,
    e o Transmissor sempre vai reenviar o pacote com seq=0.
\end{exe}

\begin{exe}{7}
    O número de sequência serve apenas para dizer:
    ``na minha sequência de blocos que
    estou enviando para você
    esse é o bloco número x''.
    Isso é usado para garantir que o
    receptor ``colecionou todas as figurinhas''.
    Para o transmissor, não interessa saber
    se ele recebeu todos os ACKs que o receptor enviou,
    basta saber que o receptor recebeu todas as ``figurinhas''.

    Mas ok, o transmissor recebeu
    o ACK número 43 referente ao bloco número 43,
    e então
    o ACK número 46 referente ao bloco número 44.
    Ele perdeu 2 ACKs, mas está confirmado que
    foram recebidos os blocos 43 e 44,
    o que o transmissor vai fazer com a informação extra
    de que 2 ACKs foram perdidos?
    Já existe um mecanismo para contornar a perda de pacotes:
    o timeout.
    Então o transmissor já sabia que provavelmente
    o pacote foi perdido.

    Uma outra possibilidade seria já ter acontecido o timeout
    (nesse caso 2 timeouts),
    mas chegou o ACK número 44 ao invés do 46 (do exemplo anterior),
    ambos referentes ao bloco 44.
    Talvez ainda chegue os ACKs 45 e 46,
    mas o transmissor vai ignorá-los de qualquer maneira.
    Imagino que a única coisa que o transmissor pode fazer
    com essa informação é ajustar o tempo do timeout,
    e isso não afeta na corretude do protocolo,
    apenas na eficiência dele.
\end{exe}

\begin{exe}{11}
    Sim para o \texttt{Wait for 1 from below},
    considerando que o pacote na variável \texttt{sndpkt}
    é o mesmo da última atribuição
    (ACK, 0/1 e checksum não mudaram
    desde a última vez que o pacote foi construído).
    Uma consideração é que o primeiro pacote recebido
    pode estar corrompido,
    então caso \texttt{sndpkt} começasse inicializado
    (com \texttt{make\_pkt(ACK, 1, checksum)})
    ou fosse garantido o primeiro pacote não está corrompido
    também funcionaria para \texttt{Wait for 0 from below}.

    Se a auto-transição fosse removida,
    no evento de um pacote corrompido,
    o receptor não transmitiria nenhum pacote
    para o transmissor,
    mas o transmissor está aguardando um pacote
    e o receptor também está aguardando um pacote:
    o sistema entraria em deadlock.
\end{exe}

\begin{exe}{12}
    No caso de reenviar o pacote
    no evento de ACK errado ou pacote corrompido,
    abriria a possibilidade do timer disparar
    e reenviar uma segunda vez,
    mesmo se for resetado no primeiro reenvio do pacote.
    O receptor receberia 2 pacotes iguais e
    responderia com 2 ACKs iguais.
    O transmissor receberia o primeiro ACK e
    enviaria o próximo pacote;
    receberia o segundo ACK e reenviaria o mesmo pacote.
    E agora está sendo transitido um mesmo bloco 2 vezes.
    A cada timeout que não fosse correspondido por uma perda,
    aumentaria em 1 na quantidade de blocos redundantes sendo
    transmitido na rede.

    O protocolo continuaria funcionando,
    mas teria a possibilidade de um efeito indesejado:
    pacotes redundantes voando na rede.
\end{exe}

\begin{exe}{13}
    Pela dificuldade em desenhar num computador,
    vou descrever a sequência de eventos e
    possívelmente incluir uma foto do diagrama depois.

    Notação:
    \begin{itemize}
        \item \(T\) é o transmissor
        \item \(R\) é o receptor
        \item \(D_ni\) é o pacote do bloco de dados \(n\),
            o \(i\)-ésimo pacote enviado por \(T\).
        \item \(A_ni\) é o pacote do ACK confirmando \(n\),
            referente ao \(i\)-ésimo pacote enviado por \(T\).
    \end{itemize}
    \par\textit{Nota 1:
        \(D_ni\) sempre vai do \(T\) para \(R\) e
        \(A_ni\) sempre vai do \(R\) para \(T\).}
    \par\textit{Nota 2:
        \(i\) não aparece no protocolo/máquinas de estado,
        por isso \(T\) e \(R\) não conseguem
        distinguir pacotes com mesmo \(n\).}

    \begin{multicols}{2}
    \begin{enumerate}
        \item Início: \(T\) envia \(D_00\)
        \item Timeout: \(T\) envia \(D_01\)
        \item Reordenação: \(R\) recebe \(D_01\), envia \(A_01\)
            \par (Nota: \(D_00\) continua na rede)
        \item Receive: \(T\) recebe \(A_01\), envia \(D_12\)
        \item Reordenação: \(R\) recebe \(D_12\), envia \(A_12\)
            \par (Nota: \(D_00\) continua na rede)
        \item Receive: \(T\) recebe \(A_12\), envia \(D_03\)
        \item Receive: \(R\) recebe \(D_00\), envia \(A_00\)
            \par (Nota: \(D_03\) continua na rede)
        \item Receive: \(R\) ignora \(D_03\), envia \(A_03\)
        \item Receive: \(T\) recebe \(A_00\), envia \(D_14\)
            \par (Nota: \(A_03\) continua na rede)
        \item Receive: \(T\) recebe \(A_03\), ignora
    \end{enumerate}
    \end{multicols}
    Note que o primeiro pacote \(D_00\) ficou
    preso na rede por muito tempo e
    no evento \(7.\) o receptor
    confunde o pacote antigo \(D_00\)
    com o que deveria ter sido recebido \(D_03\).
    Quando o pacote correto chega no evento \(8.\),
    o receptor apenas o ignora.
    Nesse exemplo, o terceiro bloco recebido estava errado.
    E nenhuma das partes conseguiu detectar isso.
\end{exe}

\begin{exe}{18}
    Pela dificuldade em desenhar num computador,
    vou descrever a sequência de eventos e
    possívelmente incluir uma foto do diagrama depois.

    Como só vão ter no máximo 2 blocos em transmisão,
    o protocolo vai ser parecido com SR com N = 2.

    As transições que começam com (*) são as transições iniciais
    (no livro são as setas pontilhadas que vêm do além).
    No final de cada ação pode ter um \texttt{goto(E)},
    que diz: ``essa transição leva para o estado E''.
    Se não tiver um \texttt{goto} no final da ação,
    considere que é uma transição para o próprio estado.

    Estados do Transmissor:
    \begin{enumerate}
        \item \texttt{Wait for 0 from above}:
            \begin{itemize}
                \item (*) \(\Lambda\) \par
                    seqnum = 0 \\
                    rcvedACK0 = false \\
                    rcvedACK1 = false
                \item \texttt{rdt\_send(data)} \par
                    sndpkt[0]
                        = make\_pkt(seqnum, 0, data) \\
                    \texttt{goto(Wait for 1 from above)}
                \item \texttt{rdt\_rcv(rcvpkt)} \par
                    \(\Lambda\)
            \end{itemize}
        \item \texttt{Wait for 1 from above} \par
            \begin{itemize}
                \item \texttt{rdt\_send(data)} \par
                    sndpkt[1]
                        = make\_pkt(seqnum, 1, data) \\
                    udt\_send(sndpkt[0]) \\
                    udt\_send(sndpkt[1]) \\
                    start\_timer \\
                    \texttt{goto(Wait for ACKs)}
                \item \texttt{rdt\_rcv(rcvpkt)} \par
                    \(\Lambda\)
            \end{itemize}
        \item \texttt{Wait for ACKs} \par
            \begin{itemize}
                \item \texttt{rdt\_rcv(rcvpkt)
                        \&\& hasseqnum(rcvpkt, seqnum)
                        \&\& isZero(rcvpkt \\\tab{4ex}
                        \&\& !rcvedACK1} \par
                    rcvedACK0 = true
                \item \texttt{rdt\_rcv(rcvpkt)
                        \&\& hasseqnum(rcvpkt, seqnum)
                        \&\& isZero(rcvpkt) \\\tab{4ex}
                        \&\& rcvedACK1} \par
                    seqnum++ \\
                    rcvedACK1 = false \\
                    stop\_timer \\
                    \texttt{goto(Wait for 0 from above)}
                \item \texttt{rdt\_rcv(rcvpkt)
                        \&\& hasseqnum(rcvpkt, seqnum)
                        \&\& isOne(rcvpkt) \\\tab{4ex}
                        \&\& !rcvedACK0} \par
                    rcvedACK1 = true
                \item \texttt{rdt\_rcv(rcvpkt)
                        \&\& hasseqnum(rcvpkt, seqnum)
                        \&\& isOne(rcvpkt) \\\tab{4ex}
                        \&\& rcvedACK0} \par
                    seqnum++ \\
                    rcvedACK0 = false \\
                    stop\_timer \\
                    \texttt{goto(Wait for 0 from above)}
                \item \texttt{timeout} \par
                    if ( !rcvedACK0 ) \\\tab{2ex}
                        udt\_send(sndpkt[0]) \\
                    if ( !rcvedACK1 ) \\\tab{2ex}
                        udt\_send(sndpkt[1]) \\
                    start\_timer
                \item \texttt{rdt\_rcv(rcvpkt)
                        \&\& !hasseqnum(rcvpkt, seqnum)} \par
                    \(\Lambda\)
            \end{itemize}
    \end{enumerate}

    Estados do Receptor:
    \begin{enumerate}
        \item \texttt{Wait for 01 from below}:
            \begin{itemize}
                \item (*) \(\Lambda\) \par
                    seqnum = 0
                \item \texttt{rdt\_rcv(rcvpkt)
                        \&\& hasseqnum(rcvpkt, seqnum)
                        \&\& isZero(rcvpkt)} \par
                    extract(rcvpkt, data) \\
                    deliver\_data(data) \\
                    sndpkt = make\_pkt(seqnum, ACK, 0) \\
                    udt\_send(sndpkt) \\
                    \texttt{goto(Wait for 1 from below)}
                \item \texttt{rdt\_rcv(rcvpkt)
                        \&\& hasseqnum(rcvpkt, seqnum)
                        \&\& isOne(rcvpkt)} \par
                    extract(rcvpkt, data) \\
                    stred\_data = data \\
                    sndpkt = make\_pkt(seqnum, ACK, 1) \\
                    udt\_send(sndpkt) \\
                    \texttt{goto(Wait for 0 from below)}
                \item \texttt{rdt\_rcv(rcvpkt)
                        \&\& !hasseqnum(rcvpkt, seqnum)} \par
                    parity = getparity(rcvpkt) \\
                    sndpkt = make\_pkt(seqnum, ACK, parity) \\
                    udt\_send(sndpkt)
            \end{itemize}
        \item \texttt{Wait for 0 from below}:
            \begin{itemize}
                \item \texttt{rdt\_rcv(rcvpkt)
                        \&\& isZero(rcvpkt)} \par
                    extract(rcvpkt, data) \\
                    deliver\_data(data) \\
                    deliver\_data(stred\_data) \\
                    sndpkt = make\_pkt(seqnum, ACK, 0) \\
                    udt\_send(sndpkt) \\
                    seqnum++ \\
                    \texttt{goto(Wait for 01 from below)}
                \item \texttt{rdt\_rcv(rcvpkt)
                        \&\& isOne(rcvpkt)} \par
                    sndpkt = make\_pkt(seqnum, ACK, 1) \\
                    udt\_send(sndpkt)
            \end{itemize}
        \item \texttt{Wait for 1 from below}:
            \begin{itemize}
                \item \texttt{rdt\_rcv(rcvpkt)
                        \&\& isOne(rcvpkt)} \par
                    extract(rcvpkt, data) \\
                    deliver\_data(data) \\
                    sndpkt = make\_pkt(seqnum, ACK, 1) \\
                    udt\_send(sndpkt) \\
                    seqnum++ \\
                    \texttt{goto(Wait for 01 from below)}
                \item \texttt{rdt\_rcv(rcvpkt)
                        \&\& isZero(rcvpkt)} \par
                    sndpkt = make\_pkt(seqnum, ACK, 0) \\
                    udt\_send(sndpkt)
            \end{itemize}
    \end{enumerate}

    Descrição adicional do protocolo:
    \begin{itemize}
        \item Pacotes do transmissor para receptor contém:
            um `seqnum'
            (número com quantidade arbitrária de bits maior que 0),
            `parity' (0 ou 1) e `data'.
        \item Pacotes do receptor para transmissor contém:
            um `seqnum'
            (número com a mesma quantidade de bits do
            que o `seqnum' do transmissor),
            `ACK' (constante mágica que identifica um ACK) e
            `parity' (0 ou 1).
        \item Os procedimentos `getparity', `isZero' e `isOne'
            interagem com `parity' do pacote, que respectivamente:
            retorna o bit de paridade,
            verifica que o bit de paridade é 0 e
            verifica que o bit de paridade é 1.
            Funcionam abstratamente para ambos os modelos de pacotes.
        \item O \texttt{!} (exclamação) que precede um boolean
            funciona como uma negação lógica
            (igual a C e outras linguagens de programação).
    \end{itemize}

    Observações adicionais:
    \begin{itemize}
        \item No transmissor,
            sería possível transformar `rcvedACK0' e `rcvedACK1'
            em estados, mas teria muito copy-paste.
            Achei que ia ficar menos legível.
        \item Nos transmissor e receptor,
            como não há reordenação de mensagens,
            o `seqnum' poderia estar nos valores de 0 e 1,
            e transformá-lo em estados simétricos,
            assim como aparece nas FSM das versões 2.2 e 3.0.
        \item No receptor,
            no estado \texttt{Wait for 01 from below},
            no evento de receber um pacote com paridade 1,
            é necessário armazenar os dados do pacote
            para que seja entregue em ordem para a camada de cima.
            É equivalente armazenar o pacote inteiro
            e depois (quando o 0 chegar)
            extrair o dado e entregar o dado.
        \item Para ``suportar'' corrupção,
            basta adicionar um checksum e
            droppar/ignorar mensagens corrompidas.
        \item Já ``suporta'' reordenação,
            desde que a quantidade de bits em `seqnum'
            seja grande o suficiente.
    \end{itemize}
\end{exe}

\begin{exe}{22}
    \begin{exe-list}
    \item
        \begin{itemize}
            \item \((k, k+1, k+2, k+3)\);
            \item \((k-1, k, k+1, k+2)\),
            \item \((k-2, k-1, k, k+1)\),
            \item \((k-3, k-2, k-1, k)\),
            \item \((k-4, k-3, k-2, k-1)\)
        \end{itemize}
        Esses são os casos em que o transmissor recebeu
        \(k, k-1, k-2, k-3, k-4\) ACKs
        (começando a contar do \(0\)), respectivamente.
        Os ACKs que faltam estão em trânsito ou foram perdidos
        (note que os ACKs são acumulativos,
        então receber o ACK do \(k-1\) vale também como
        ACK do \(k-2\)).

        O receptor já recebeu todos os pacotes que vieram
        antes do \(k\).

        Os pacotes com número de sequência maior ou igual à \(k\)
        já foram enviados pelo transmissor
        ou ainda não existem
        (o transmissor espera pela camada de cima).
    \item
        Considerando que o timeout só ocorre quando
        um pacote que o transmissor envia é perdido,
        os valores de ACK que podem estar à caminho
        do transmissor são
        maiores ou iguais ao menor número da sequência
        (vamos chamar de \(x\))
        e menores que \(k\).

        Suponha que existe um valor de ACK menor que \(x\)
        no ``meio do caminho'' para o transmissor,
        o transmissor já confirmou esse pacote.
        Então, ou isso é um ACK duplicado
        (que nesse caso não conta porque eu pedi isso)
        ou ele foi reordenado e outro ACK maior
        chegou antes (confirmando o menor).
        Logo, ou o ACK é maior ou igual a \(x\)
        ou teve um ``timeout ruim''.

        Suponha que existe um valor de ACK maior ou igual à \(k\)
        no ``meio do caminho'' para o transmissor,
        o receptor ainda está esperando esse pacote,
        então ele não poderia ter enviado.
        Logo, o valor de ACK tem que ser menor que \(k\).

        Concluindo, o valor de ACK ``em trânsito''
        está no intervalo de \([x, k)\) ou
        é um ACK duplicado.
    \end{exe-list}
\end{exe}

\begin{exe}{24}
    \begin{exe-list}
    \item Sim, ele pode receber um ACK duplicado
        ``mais antigo'' que a janela atual,
        assumindo possibilidade de perda.
    \item Sim, ele pode receber um ACK duplicado
        ``mais antigo'' que a janela atual,
        assumindo possibilidade de perda.
        Mas também pode receber um ACK ``atrasado'',
        assumindo possibilidade de reordenação.
    \item Sim, GBN e SR são equivalentes quando a janela é 1.
        Quando GBN tem janela 1,
        ele tem um timer para cada pacote da janela.
        Quando SR tem janela 1,
        ele reenvia todos os pacotes quando ocorre um timeout.
        (Veja item a baixo).
    \item Sim, GBN envia todos os (1) pacotes e
        espera algum ACK (o do pacote enviado)
        chegar para avançar a janela
        (liberar o próximo pacote).
        Tem timeout.
    \end{exe-list}
\end{exe}

\begin{exe}{31}
    \begin{itemize}
        \item \textbf{EstimatedRTT:}
            \begin{align*}
                EstimatedRTT_{i+1} &:= (1 - \alpha) \; EstimatedRTT_i
                    + \alpha \; SampleRTT_{i+1} \\
                EstimatedRTT_k &= 100 \ms \\
                EstimatedRTT_{k+1} &= 0.875 \times 100 \ms
                    + 0.125 \times 106 \ms \\
                &= 100.75 \ms \\
                EstimatedRTT_{k+2} &= 0.875 \times 100.75 \ms
                    + 0.125 \times 120 \ms \\
                &= 103.156 \ms \\
                EstimatedRTT_{k+3} &= 0.875 \times 100.75 \ms
                    + 0.125 \times 140 \ms \\
                &= 107.762 \ms \\
                EstimatedRTT_{k+4} &= 0.875 \times 100.75 \ms
                    + 0.125 \times 90 \ms \\
                &= 105.542 \ms \\
                EstimatedRTT_{k+5} &= 0.875 \times 100.75 \ms
                    + 0.125 \times 115 \ms \\
                &= 106.724 \ms
            \end{align*}
        \item \textbf{DevRTT:}
            \begin{align*}
                DevRTT_{i+1} &:= (1 - \beta) \; DevRTT_i
                    + \beta \; |SampleRTT_{i+1} - EstimatedRTT_{i+1}| \\
                DevRTT_k &= 5 \\
                DevRTT_{k+1} &= 0.75 \times 5 \ms
                    + 0.25 \times |106 - 100| \ms \\
                &= 5.25 \ms \\
                DevRTT_{k+2} &= 0.75 \times 5.25 \ms
                    + 0.25 \times |120 - 100.75| \ms \\
                &= 8.75 \ms \\
                DevRTT_{k+3} &= 0.75 \times 8.75 \ms
                    + 0.25 \times |140 - 103.156| \ms \\
                &= 15.7735 \ms \\
                DevRTT_{k+4} &= 0.75 \times 15.7735 \ms
                    + 0.25 \times |90 - 107.762| \ms \\
                &= 16.2706 \ms \\
                DevRTT_{k+5} &= 0.75 \times 16.2706 \ms
                    + 0.25 \times |115 - 105.542| \ms \\
                &= 14.5675 \ms
            \end{align*}
        \item \textbf{TimeoutInterval:}
            \begin{align*}
                TI_i &:= EstimatedRTT_i + 4 \; DevRTT_i \\
                TI_k &= 100 + 4 \times 5 \\
                &= 120 \ms \\
                TI_{k+1} &= 100.75 + 4 \times 5.25 \\
                &= 121.75 \ms \\
                TI_{k+2} &= 103.156 + 4 \times 8.75 \\
                &= 138.156 \ms \\
                TI_{k+3} &= 107.762 + 4 \times 15.7735 \\
                &= 170.856 \ms \\
                TI_{k+4} &= 105.542 + 4 \times 16.2706 \\
                &= 170.624 \ms \\
                TI_{k+5} &= 106.724 + 4 \times 14.5675 \\
                &= 164.994 \ms
            \end{align*}
    \end{itemize}
\end{exe}

\begin{exe}[Questão]{10}
    \begin{exe-list}
    \item
        \begin{quote}
            The Chebyshev Bound dictates that
            the probability for a random value
            exceeding a range around a mena
            depends on the standard deviation.
            Jacobson's Algorithm uses the Chebyshev Bound
            to produce reasonable retransmition time-out (RTO) values.
        \end{quote}
        A primeira frase diz (algo mais geral) que
        se temos a esperança e variância
        dos round-trip times (RTT),
        podemos saber quanto tempo o RTT vai demorar
        para ser um evento com probabilidade menor que \(\varepsilon\)
        (um número tão pequeno que ``nunca'' vai acontecer).
        Exemplo: ``nunca'' vai acontecer de um RTT
        demorar mais que \(3\) segundos.

        A segunda disse que podemos usar isso
        para criar uma boa estimativa de um RTO,
        um tempo escolhido para o timeout.
    \item
        O algoritmo de Jacobson é rápido mas tem alguns problemas:
        é pouco preciso;
        RTO cresce muito rápido quando ocorre timeouts;
        quando o pacote é perdido, fica muito tempo esperando;
        e a variância é muito instável,
        gerando grandes picos no RTO calculado.

        A ideia é usar a mesma estrutura do algoritmo,
        mas alterar a forma de previsão do RTT,
        nesse caso usando Redes Neutrais.
        E estimando melhor o RTT seria possível
        detectar perdas mais cedo e
        ao mesmo tempo
        evitar o reenvio desnecessário de segmentos.

        Usando dois modelos diferentes de Redes Neurais,
        um se saiu consideravelmente melhor e
        o outro foi levemente melhor,
        no quesito quantidade de segmentos retransmitidos.
    \end{exe-list}
\end{exe}

\begin{exe}{37}
    \begin{exe-list}
    \item
        Assumindo que nenhum pacote foi perdido
        (além do enunciado),
        sem corrupção e reordenação.
        \begin{itemize}
            \item \textbf{GBN:}
                \begin{enumerate}
                    \item A envia pacotes \(0 \dots 4\)
                    \item Pacote \(1\) é perdido
                    \item B recebe pacotes \(0, 2 \dots 4\),
                        envia ACK \(0\), ignora os outros
                    \item A recebe ACK \(0\)
                    \item Timeout, A reenvia pacotes \(1 \dots 4\)
                    \item B recebe pacotes \(1 \dots 4\),
                        envia ACK \(1 \dots 4\)
                \end{enumerate}
            \item \textbf{SR:}
                \begin{enumerate}
                    \item A envia pacotes \(0 \dots 4\)
                    \item Pacote \(1\) é perdido
                    \item B recebe pacotes \(0, 2 \dots 4\),
                        envia ACKs \(0, 2 \dots 4\)
                    \item A recebe ACKs \(0, 2 \dots 4\)
                    \item Timeout, A reenvia pacote \(1\)
                    \item B recebe pacote \(1\),
                        envia ACK \(1\)
                \end{enumerate}
            \item \textbf{TCP:}
                \begin{enumerate}
                    \item A envia pacotes \(0 \dots 4\)
                    \item Pacote \(1\) é perdido
                    \item B recebe pacotes \(0, 2 \dots 4\),
                        envia ACK \(0\) para cada pacote recebido
                        (total 4)
                    \item A recebe 4 ACKs \(0\),
                        reenvia pacote \(1\)
                        (depois de receber o terceiro)
                    \item B recebe pacote \(1\),
                        envia ACK \(4\)
                \end{enumerate}
        \end{itemize}
        Resumo:
        {\par\centering\begin{tabular}{|l|c|c|c|c|}\hline
            & \textbf{Pacotes enviados} & \textbf{Total (pacotes)}
                & \textbf{ACKs enviados} & \textbf{Total (ACKs)}
                \\\hline
            \textbf{GBN} & 0,1,2,3,4,1,2,3,4 & 9 & 0,1,2,3,4 & 5
                \\\hline
            \textbf{SR}  & 0,1,2,3,4,1 & 6 & 0,2,3,4,1 & 5 \\\hline
            \textbf{TCP} & 0,1,2,3,4,1 & 6 & 0,0,0,0,4 & 5 \\\hline
        \end{tabular}\medskip}
        Note que os ACKs no TCP ``de verdade'' deveriam ter sido
        incrementados em 1 (pedindo o próximo byte),
        mas deixei assim para ficar mais fácil de comparar
        com os outros.
    \item TCP, pois não precisou do timeout.
    \end{exe-list}
\end{exe}

\begin{exe}{40}
    \begin{exe-list}
    \item \([1, 5)\) e \([23, \_)\),
        são os intervalos que têm um crescimento exponencial.
    \item \([5, 22)\),
        é o intervalo que tem um crescimento linear.
    \item ACK triplicado, pois não voltou para o slow start.
    \item Timeout, pois voltou para o slow start.
    \item Um valor no intervalo \((16, 32]\),
        pois é quando para de crescer exponencialmente.
    \item \(\frac{42}{2} = 21\),
        foi settado na ultima falha (em 16)
    \item \(\frac{29}{2} = 14\),
        pois o último falha ocorreu com o tamanho de janela
        igual à 29.
    \item Assumindo que não vai ter mais falhas
        (ACKs triplicados e timeouts).
        O slow start termina em
        \(23 + \lceil \log_2 14 \rceil = 23 + 4 = 27\)
        e a partir daí é incrementado em \(1\),
        então o tamanho da janela vai ser
        \(16 + (70 - 27) = 59\).
    \item O tamanho da janela vai ser \(\frac82 + 3 = 7\) e
        o thresh hold vai ser \(\frac82 = 4\).
    \item O tamanho da janela vai ser \(2^{(19-16+1)} = 2^{2} = 4\) e
        o thresh hold vai ser \(\frac{42}{2} = 21\).
    \item \(1, 2, 4, 8, 16, 32\) nos rounds \(17\) até \(22\),
        totalizando em \(63\).
    \end{exe-list}
\end{exe}

\begin{exe}{45}
    TODO
    \begin{exe-list}
    \item
    \item
    \end{exe-list}
\end{exe}

\begin{exe}{49}
    TODO
\end{exe}

\begin{exe}{52}
    TODO
\end{exe}

\end{document}
