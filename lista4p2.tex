\documentclass{article}

\usepackage[width=14cm, left=3cm, top=2cm]{geometry}

\usepackage[brazil]{babel}
\usepackage[T1]{fontenc}
\usepackage[utf8]{inputenc}

\usepackage{amsmath}
\usepackage{amssymb}

\usepackage{multicol}

\usepackage{multirow}

\usepackage{ifthen}
%\usepackage{minted}

\newcommand{\blank}{\rule[0pt]{5em}{.3pt}}
\newcommand{\nobody}{ninguém}
\newcommand{\preamble}[2]{\noindent%
    Fiz esse trabalho com a ajuda de {\bfseries #1}
    e consultei {\bfseries #2}.
    A versão final do trabalho foi feita
    por mim de forma independente.
    Respostas sem no mínimo 3 frases de justificativa
    não contam ponto.
    \par\noindent Assinatura: \blank\blank\bigskip}

\newcounter{exe-list}
\newenvironment{exe-list}
    {\begin{list}{\alph{exe-list}.}{\usecounter{exe-list}}}
    {\end{list}}

\newenvironment{exe}[2][Problema]
    {\newcommand{\opt}{(Opcional)}%
    \newcommand{\sketch}[1]{{\bfseries Rascunho:} ##1}%
    \medskip\par\noindent\ifthenelse{\equal{#1}{}}
        {\textbf{\large #2}}
        {\textbf{\large #1~#2}}%
    \medskip\par\noindent}
    {\medskip}

\title{Redes de Computadores - Lista 4 (parte 2)}
\author{Daniel Kiyoshi Hashimoto Vouzella de Andrade - 119025937}
\date{Novembro 2022}

\begin{document}
\maketitle

%\preamble{\blank}{\blank}

\begin{exe}{16}
    \begin{exe-list}
    \item
        \begin{itemize}
        \item Host 1: 192.168.1.1
        \item Host 2: 192.168.1.2
        \item Host 3: 192.168.1.3
        \end{itemize}
    \item \hspace{1ex}\par
        \begin{center} \begin{tabular}{|lr|lr|} \hline
            \multicolumn{4}{|c|}{\textbf{NAT Translation Table}} \\\hline
            \multicolumn{2}{|c|}{WAN side}
                & \multicolumn{2}{|c|}{LAN side} \\\hline
            24.34.112.253 & R1 & 192.168.1.1 & A1 \\\hline
            24.34.112.253 & R2 & 192.168.1.1 & A2 \\\hline
            24.34.112.253 & R3 & 192.168.1.2 & B1 \\\hline
            24.34.112.253 & R4 & 192.168.1.2 & B2 \\\hline
            24.34.112.253 & R5 & 192.168.1.3 & C1 \\\hline
            24.34.112.253 & R6 & 192.168.1.3 & C2 \\\hline
        \end{tabular} \end{center}
        Onde, R1, R2, R3, R4, R5, R6, A1, A2, B1, B2, C1 e C2
        são ports arbitrários escolhidos pela letra
        (R significa que o roteador escolheu, \dots).
    \end{exe-list}
\end{exe}

\begin{exe}{17}
    \begin{exe-list}
    \item
        Assumindo que o primeiro número de sequência gerado
        não se altera caso seja criada uma nova conexão,
        podemos contar quantas sequências distintas existem.
        Isso é apenas uma aproximação, pois
        é possível que
        existam dois hosts com número de sequência muito próximos,
        e assim, confundir a contagem.
        Além disso, não é possivel detectar um ``host silencioso''
        (que não gera tráfego para fora da rede).
    \item
        Assumindo que não temos acesso ao gerador de números aleatório
        dos hosts, não funciona, pois não podemos
        relacionar os números de sequência com os hosts.
        Caso contrário (temos acesso aos RNGs),
        podemos atribuir uma relação entre dois números de sequência,
        e ``volta'' a funcionar (embora um pouco mais complicada
        e com a possibilidade de ser menos preciso).
    \end{exe-list}
\end{exe}

\begin{exe}{18}
    Existe uma estratégia conhecida como \emph{Hole Punching},
    entretando, a princípio é muito difícil fazer isso,
    pois o NAT não permite que alguém ``de fora''
    acesse alguém ``de dentro''.
    Dessa forma, Arnold não consegue acessar Bernard,
    nem Bernard consegue acessar Arnold.
    Uma possível solução é configurar o roteador de um dos dois,
    para deixar hardcoded uma porta para o host (Arnold, por exemplo).

    A ideia do \emph{Hole punching} é
    usar a ajuda de um ``servidor público''
    (público pois precisa estar não usar um NAT).
    Ambos os hosts (Arnold e Bernard) vão, inicialmente,
    ``se registrar'' enviando uma mensagem para o servidor.
    Ao enviar essa mensagem, o roteador vai atualizar
    a tabela do NAT e o servidor vai saber o IP e port dos hosts.
    Agora o servidor envia o par IP,port de um host para o outro host,
    e assim conseguimos fazer o \emph{Hole punching}.
\end{exe}

\begin{exe}{19}
    \begin{center} \begin{tabular}{|c|c|} \hline
        \multicolumn{2}{|c|}{\textbf{s2 Flow Table}} \\\hline
        \textbf{Match} & \textbf{Action} \\\hline
        In Port=1 ; IP Src=10.3.*.* ; IP Dst=10.1.*.*
            & Forward(2) \\\hline
        In Port=2 ; IP Src=10.1.*.* ; IP Dst=10.3.*.*
            & Forward(1) \\\hline
        IP Dst=10.2.0.3
            & Forward(3) \\\hline
        IP Dst=10.2.0.4
            & Forward(4) \\\hline
    \end{tabular} \end{center}
\end{exe}

\begin{exe}{21}
    \begin{center} \begin{tabular}{|c|c|} \hline
        \multicolumn{2}{|c|}{\textbf{s1 Flow Table}} \\\hline
        \textbf{Match} & \textbf{Action} \\\hline
        In Port=4 ; IP Src=10.2.*.* ; IP Dst=10.3.*.*
            & Forward(1) \\\hline
        IP Dst=10.1.0.1
            & Forward(2) \\\hline
        IP Dst=10.1.0.2
            & Forward(3) \\\hline
    \end{tabular} \end{center}
    \begin{center} \begin{tabular}{|c|c|} \hline
        \multicolumn{2}{|c|}{\textbf{s3 Flow Table}} \\\hline
        \textbf{Match} & \textbf{Action} \\\hline
        IP Dst=10.3.0.6
            & Forward(1) \\\hline
        IP Dst=10.3.0.5
            & Forward(2) \\\hline
        In Port=4 ; IP Src=10.2.*.* ; IP Dst=10.1.*.*
            & Forward(3) \\\hline
    \end{tabular} \end{center}
\end{exe}

\end{document}
