\documentclass{article}

\usepackage[width=14cm, left=3cm, top=3cm]{geometry}

\usepackage[brazil]{babel}
\usepackage[T1]{fontenc}
\usepackage[utf8]{inputenc}

\usepackage{amsmath}
\usepackage{amssymb}

\usepackage{multicol}

\usepackage{hyperref}

\usepackage{ifthen}
%\usepackage{minted}

\newcommand{\To}{\Longrightarrow}

\newcommand{\blank}{\rule[0pt]{5em}{.3pt}}
\newcommand{\nobody}{ninguém}
\newcommand{\preamble}[2]{\noindent%
    Fiz esse trabalho com a ajuda de {\bfseries #1}
    e consultei {\bfseries #2}.
    A versão final do trabalho foi feita
    por mim de forma independente.
    \par\noindent Assinatura: \blank\blank\bigskip}

\newcounter{exe-list}
\newenvironment{exe-list}
    {\begin{list}{\alph{exe-list}.}{\usecounter{exe-list}}}
    {\end{list}}

\newenvironment{exe}[2][Problema]
    {\newcommand{\opt}{(Opcional)}%
    \newcommand{\sketch}[1]{{\bfseries Rascunho:} ##1}%
    \medskip\par\noindent\ifthenelse{\equal{#1}{}}
        {\textbf{\large #2}}
        {\textbf{\large #1~#2}}%
    \medskip\par\noindent}
    {\medskip}

\newcommand{\Declare}[3]{\DeclareMathOperator{#1}{#2{}#3}}
\newcommand{\DeclareBit}[2]{\Declare{#1}{#2}{b}}
\DeclareBit{\bit}{}
\DeclareBit{\kbit}{k}
\DeclareBit{\Mbit}{M}
\DeclareBit{\Gbit}{G}
\newcommand{\DeclareByte}[2]{\Declare{#1}{#2}{byte}}
\DeclareByte{\byte}{}
\DeclareByte{\kbyte}{k}
\DeclareByte{\Mbyte}{M}
\DeclareByte{\Gbyte}{G}
\newcommand{\DeclareBps}[2]{\Declare{#1}{#2}{bps}}
\DeclareBps{\bps}{}
\DeclareBps{\kbps}{k}
\DeclareBps{\Mbps}{M}
\DeclareBps{\Gbps}{G}

\title{Redes de Computadores - Lista 1 (parte 2)}
\author{Daniel Kiyoshi Hashimoto Vouzella de Andrade - 119025937}
\date{Setembro 2022}

\begin{document}
\maketitle

\preamble{\nobody}{\nobody}

\begin{exe}{12}
    \begin{itemize}
    \item Tempo para um pacote ser transmitido:
        \( \frac{L}{R}
            = \frac{1500 \byte}{2 \Mbps}
            = \frac{1500 \cdot 2^8}{2 \cdot 1000000}
            = 0.192 \)
        \item Tempo para esvaziar a fila (queue delay):
        \( 4.5 \cdot .0192 = 0.864 \)
    \item Bits para transmitir (genérico):
        \( L \; n + x \)
    \item Queue delay (genérico):
        \( \frac{L \; n + x}{R} \)
    \end{itemize}
\end{exe}

\begin{exe}{13}
    \begin{exe-list}
    \item \begin{itemize}
            \item Queue delay para o pacote \(i\)
                (contando a partir do \(0\)):
                \( i \; \frac{L}{R} \)
            \item Média do queue delay:
                \[
                    \frac1N \sum_{i = 0}^{N-1} i \; \frac{L}{R}
                    = \frac{L}{N \; R} \sum_{i = 0}^{N-1} i
                    = \frac{L}{N \; R} \; \frac{N \; (N-1)}{2}
                    = \frac{L \; (N-1)}{R}
                \]
        \end{itemize}
    \item \begin{itemize}
            \item Tempo para a fila esvaziar:
                \( N \; \frac{L}{R} \)
            \item A fila fica vazia na próxima chegada,
                então o tempo médio é o mesmo da questão anterior:
                \( \frac{L \; (N-1)}{R} \)
        \end{itemize}
    \end{exe-list}
\end{exe}

\begin{exe}{14}
    \begin{exe-list}
    \item \begin{itemize}
            \item Total delay (\( d_{queue} + d_{trans} \)):
                \[
                    \frac{I \; L}{(1 - I) \; R} + \frac{L}{R}
                    = \frac{L}{R} \; \left(
                            \frac{I}{1 - I} + 1
                        \right)
                    = \frac{L}{R} \; \left(
                            \frac{I + 1 - I}{1 - I}
                        \right)
                    = \frac{L}{R} \; \frac{1}{1 - I}
                    = \frac{L}{R \; (1 - I)}
                \]
            \item Substituindo (\( I = \frac{L \; a}{R} \)):
                \[
                    \frac{L}{R \; (1 - \frac{L \; a}{R})}
                    = \frac{L}{R - L \; a}
                \]
        \end{itemize}
    \item \begin{itemize}
            \item Troca de variáveis \( x := \frac{L}{R} \):
                \[
                    \frac{L}{R \; (1 - \frac{L \; a}{R})}
                    = \frac{L}{R} \; \frac{1}{1 - \frac{L \; a}{R}}
                    = x \; \frac{1}{1 - a \; x}
                    = \frac{x}{1 - a \; x}
                \]
            \item \textit{Lembrete:} \(a \; x < 1\)
            \item \textit{Lembrete:} o eixo y representa o delay
            \item Não me sinto confortável imbutindo uma imagem,
                então um link: \par
                \href{https://www.desmos.com/calculator/n7ug34enga}
                {desmos.com/calculator/n7ug34enga}
            \item O gráfico faz sentido, intuitivamente:
            \begin{itemize}
                \item Quanto maior o tamanho do pacote
                    em relação à taxa de transmissão
                    (\(x := \frac{L}{R}\)),
                    maior o tempo de espera.
                \item É ``linear'' quando
                    a taxa de chegada de pacotes (\(a\))
                    é muito próxima de 0;
                    é como se só chegasse um pacote por vez,
                    então o delay é só enviar o pacote.
                \item Explode ``exponencialmente'' quando
                    a taxa de chegada é muito grande;
                    é como se chegasse mais pacotes do que
                    o sistema pode aguentar,
                    gerando uma fila de espera gigantesca.
            \end{itemize}
        \end{itemize}
    \end{exe-list}
\end{exe}

\begin{exe}{15}
    \begin{itemize}
        \item \(x\) são quantos segundos demora
            para enviar um pacote,
            então \( \frac1x \) são
            quantos pacotes são enviados por segundo
        \item Usando \(\mu := \frac{1}{x} \):
            \[
                \frac{x}{1 - a \; x}
                = \frac{\frac{x}{x}}{\frac1x - \frac{a \; x}{x}}
                = \frac{1}{\frac1x - a}
                = \frac{1}{\mu - a}
            \]
    \end{itemize}
\end{exe}

\begin{exe}{20}
    \begin{itemize}
        \item Assumindo que os \(M\) clientes-servidores estão
            usando a rede simultâneamente,
            a taxa de transmissão do ``bottleneck'':
            \( \frac{R}{M} \)
        \item Throughput percebido por cada cliente-servidor:
            \( \min\left( R_s, R_c, \frac{R}{M} \right) \)
    \end{itemize}
\end{exe}

\begin{exe}{21}
    \begin{itemize}
        \item Assumindo que só pode usar um caminho,
            mais especificamente o caminho \( 1 \le k \le M \):
            \[
                \min\left(
                    R_{(1,k)}, R_{(2,k)}, \dots,
                    R_{(N-1,k)}, R_{(N,k)} \right)
            \]
        \item Assumindo que pode usar todos os caminhos.
        \begin{itemize}
            \item Definindo \( T_k \) como o throughput
                do caminho \(k\),
                obtido no item anterior:
                \[
                    T_k := \min\left(
                        R_{(1,k)}, R_{(2,k)}, \dots,
                        R_{(N-1,k)}, R_{(N,k)} \right)
                \]
            \item Throughput total:
                \( \sum_{k = 1}^M T_k \)
        \end{itemize}
    \end{itemize}
\end{exe}

\begin{exe}[]{Multipath TCP}
    Parece ser uma implementação do conceito levantado
    na última questão:
    aproveitar os vários caminhos
    para alcançar um throuput melhor.
    Este funciona na camada de transporte.
\end{exe}

\begin{exe}{22 \opt}
    \begin{itemize}
        \item Para o pacote chegar no destino,
            precisamos de \(N\) ``falhas'',
            ou seja, \(N\) pacotes \textbf{NÃO} perdidos:
            \( (1-p)^N \)
        \item \( q := (1-p)^N \) é a probabilidade de sucesso.
            Quantas vezes temos que tentar, na média,
            para que tenhamos o sucesso?
            Isso é a esperança de uma geométrica:
            \[
                \sum_{i = i}^{+\infty} (1-q)^i \; q
                = \frac{1}{q}
                = \frac{1}{(1-p)^N}
                = (1-p)^{-N}
            \]
    \end{itemize}
\end{exe}

\begin{exe}{23}
    \begin{exe-list}
    \item A diferença dos tempos de chegada é
        a diferença dos tempos de transmissão e propagação
        do bobbleneck e do outro link:
        \( \frac{L}{R_s} - \frac{L}{R_c} \)
    \item \begin{itemize}
            \item Não é possível,
                pois o primeiro link será mais rápido que o segundo,
                então o segundo pacote irá esperar o primeiro
                que ainda estaria sendo transmitido pelo segundo link.
            \item \(T\) deve ser no mínimo o tempo que
                o segundo pacote espera no caso anterior:
                \( \frac{L}{R_c} - \frac{L}{R_s} \)
        \end{itemize}
    \end{exe-list}
\end{exe}

\begin{exe}[]{Packet Pairs}
    Uma grande vantagem é não causar
    um grande distúrbio na rede,
    ou seja, ainda é possível usar a rede
    enquanto o teste está sendo feito,
    por outro lado a estratégia de \emph{Packet Pairs}
    causa um disturbio consideravelmente menor.

    A técnica funciona enviando dois pacotes semelhantes
    com uma diferença de tempo muito pequena e
    observando a diferença de tempo de chegada dos pacotes no destino.
    Geralmente, quanto maior esse tempo (\emph{inter-time arrival})
    maior o bottleneck,
    mas existem alguns eventos que mascaram o bottleneck
    (tanto para mais quanto para menos) e
    foram desenvolvidas outras técnicas para filtrá-los.
\end{exe}

\begin{exe}{25}
    \begin{exe-list}
    \item \[
            R \cdot d_{prop}
            = 2 \Mbps \cdot \frac{2 \cdot 10^7}{2.5 \cdot 10^8}
            = 2 \cdot 10^6 \cdot 0.08
            = 0.16 \cdot 10^6
            = 160000
        \]
    \item O número máximo de bits é o produto do item anterior
        \( R \cdot d_{prop} = 160000 \)
    \item Bandwidth-delay representa a quantidade máxima de bits
        que podem ficar no link
        (não está nem no inicio nem no destino,
        está ``no meio do caminho'').

        Esse número é importante para protocolos
        com a característica de ``esperar uma resposta''
        para enviar a ``próxima parte''.
        Quando o produto é muito alto
        comparado com o tamanho do pacote,
        isso indica que um tempo considerável da rede
        está ociosa (assumindo apenas essa conexão).
        Dessa forma seria melhor mandar mais ``partes''
        antes de ``esperar uma resposta''.
    \item ``Cabem'' \( 160000 \) bits
        no link de \( 2 \cdot 10^7 \) metros.
        Então um bit ocupa
        \( \frac{16 \cdot 10^4}{2 \cdot 10^7} = 8 \cdot 10^{-3} \)
        metros, ou \( 8 \) milímeros.
    \item \begin{itemize}
            \item Bandwidth-delay product:
                \( R \; \frac{m}{s} = \frac{R \; m}{s} \)
            \item ``Tamanho'' do bit:
                \( \frac{\frac{R \; m}{s}}{m} = \frac{R}{s} \)
        \end{itemize}
    \end{exe-list}
\end{exe}

\begin{exe}{31 \opt}
    \begin{exe-list}
    \item \begin{itemize}
            \item Tempo para enviar para o primeiro switch:
                \( \frac{8 \cdot 10^6}{2 \Mbps} = 4 \)
            \item Tempo para enviar para o segundo switch (igual):
                \( \frac{8 \cdot 10^6}{2 \Mbps} = 4 \)
            \item Tempo para enviar para o destino (igual):
                \( \frac{8 \cdot 10^6}{2 \Mbps} = 4 \)
            \item Tempo total:
                \( 3 \cdot 4 = 12 \)
        \end{itemize}
    \item \begin{itemize}
            \item Tempo para enviar o primeiro pacote
                para o primeiro switch:
                \[
                    \frac{10^4}{2 \Mbps} = 0.5 \cdot 10^{-2}
                    = 0.005
                \]
            \item Tempo no qual o segundo pacote
                é recebido pelo primeiro switch:
                \( 2 \cdot 0.005 = 0.01 \)
        \end{itemize}
    \item \begin{itemize}
            \item Tempo para enviar a mensagem toda:
                \(
                    0.005 \; (800 - 1 + 3)
                    = 4 - 0.005 + 0.015
                    = 4.01
                \)
            \item Esse resultado é da ordem de \(3\) vezes melhor,
                pois todos os link (são 3)
                estavam sendo usados, aproximadamente,
                ao mesmo tempo.
                Enquanto que, no item (a),
                cada link foi usado um depois do outro.

                Essa ideia (\emph{pipeline}) foi discutida em aula.
        \end{itemize}
    \item O tamanho das mensagens diminui,
        aumentando a probabilidade dela caber no buffer
        e não ser \emph{droppada}.
        Também permite enviar pacotes por vários caminhos diferentes
        (usando mais links ``ao mesmo tempo'');
        isso, além de diminuir ainda mais o \emph{delay},
        ajuda a reduzir a perda de pacotes.
    \item Uma desvantagem é usar todos os links,
        isso pode congestionar a rede e torná-la mais próxima
        de uma circuit-switched para outras pessoas usando a rede
        (quem envia os pacotes acaba ``reservando'' uma linha).
    \end{exe-list}
\end{exe}

\begin{exe}{33 \opt}
    \begin{itemize}
        \item Tempo de um pacote em um link:
            \( \frac{L}{R} = \frac{80 + S}{R} \)
        \item Tempo total usando \( n = \frac{F}{S} \):
            \[
                \frac{L}{R} \; (n - 1 + 3)
                = \frac{L}{R} \; (n + 2)
                = \frac{80 + S}{R} \; (n + 2)
                = \frac{80 \; n + 160 + S \; n + 2 \; S}{R}
            \]
        \item Substituindo \( n := \frac{F}{S} \):
            \[
                \frac{80 \; n + 160 + S \; n + 2 \; S}{R}
                = \frac{80 \; \frac{F}{S} + 2 \cdot 80
                    + S \; \frac{F}{S} + 2 \; S}{R}
                = \frac{80 \; F + (160 + F) \; S
                    + 2 \; S^2}{R \; S}
            \]
        \item \sketch{falta minimizar a função
            para achar o melhor \(S\).}
    \end{itemize}
\end{exe}

\end{document}
