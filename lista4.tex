\documentclass{article}

\usepackage[width=14cm, left=3cm, top=2cm]{geometry}

\usepackage[brazil]{babel}
\usepackage[T1]{fontenc}
\usepackage[utf8]{inputenc}

\usepackage{amsmath}
\usepackage{amssymb}

\usepackage{multicol}

\usepackage{multirow}

\usepackage{ifthen}
%\usepackage{minted}

\newcommand{\blank}{\rule[0pt]{5em}{.3pt}}
\newcommand{\nobody}{ninguém}
\newcommand{\preamble}[2]{\noindent%
    Fiz esse trabalho com a ajuda de {\bfseries #1}
    e consultei {\bfseries #2}.
    A versão final do trabalho foi feita
    por mim de forma independente.
    Respostas sem no mínimo 3 frases de justificativa
    não contam ponto.
    \par\noindent Assinatura: \blank\blank\bigskip}

\newcounter{exe-list}
\newenvironment{exe-list}
    {\begin{list}{\alph{exe-list}.}{\usecounter{exe-list}}}
    {\end{list}}

\newenvironment{exe}[2][Problema]
    {\newcommand{\opt}{(Opcional)}%
    \newcommand{\sketch}[1]{{\bfseries Rascunho:} ##1}%
    \medskip\par\noindent\ifthenelse{\equal{#1}{}}
        {\textbf{\large #2}}
        {\textbf{\large #1~#2}}%
    \medskip\par\noindent}
    {\medskip}

\title{Redes de Computadores - Lista 4 (parte 1)}
\author{Daniel Kiyoshi Hashimoto Vouzella de Andrade - 119025937}
\date{Outubro 2022}

\begin{document}
\maketitle

%\preamble{\blank}{\blank}

\begin{exe}{1}
    \begin{exe-list}
    \item A tabela deve conter a linha:
        \begin{center} \begin{tabular}{|c|c|} \hline
            \textbf{Destination Address}
                & \textbf{Link Interface} \\\hline
            H3 & 3 \\\hline
        \end{tabular} \end{center}
    \item A princípio não, pois as tabelas não possuem uma entrada
        para a origem do pacote.
        Mas é possível, usando SDNs, criar tabelas complexas
        que utilizam o IP de origem do pacote para saber
        como tratá-lo.
    \end{exe-list}
\end{exe}

\begin{exe}{2}
    \begin{exe-list}
    \item Não, pois o \emph{shared bus} fica ``ocupado''.
        Apenas um pacote pode ser transmitido por vez.
    \item Não, apenas uma posição de memória
        pode ser escrita por vez.
    \item Não, cada port só pode receber um pacote por vez.
        Mas poderia se fossem para dois output ports diferentes.
        Utilizando arquiteturas mais sofisticadas,
        é possível ter várias ``linhas'' de comunicação,
        permitindo a transferência paralela.
    \end{exe-list}
\end{exe}

\begin{exe}{5}
    \begin{exe-list}
    \item \hspace{1ex}
        \begin{center} \begin{tabular}{|ll|c|} \hline
            \multicolumn{2}{|c|}{\textbf{Prefix}}
                & \textbf{Link Interface} \\\hline\hline
            1110.0000 & 00        & 0 \\\hline
            1110.0000 & 0100.0000 & 1 \\\hline
            1110.0000 &           & 2 \\\hline
            1110.0001 & 0         & 2 \\\hline
            \multicolumn{2}{|c|}{Otherwise} & 3 \\\hline
        \end{tabular} \end{center}
    \item
        \begin{itemize}
        \item \textbf{1100.1000 1001.0001 0101.0001 0101.0101:}\par
            Não dá match com nenhum prefixo,
            então cai no \emph{otherwise};
            escolhe 3.
        \item \textbf{1110.0001 0100.0000 1100.0011 0011.1100:}\par
            Apenas dá match no prefixo \textbf{1110.0001 0};
            escolhe 2.
        \item \textbf{1110.0001 1000.0000 0001.0001 0111.0111:}\par
            Não dá match com nenhum prefixo,
            então cai no \emph{otherwise};
            escolhe 3.
        \end{itemize}
    \end{exe-list}
\end{exe}

\begin{exe}{6}
    \begin{center} \begin{tabular}{|l|c|c|} \hline
        \textbf{Prefix} & \textbf{Range begin}
            & \textbf{Range end} \\\hline\hline
        00  & 0000.0000 & 0011.1111 \\\hline
        010 & 0100.0000 & 0101.1111 \\\hline
        011 & 0110.0000 & 0111.1111 \\\hline
        10  & 1000.0000 & 1011.1111 \\\hline
        11  & 1100.0000 & 1111.1111 \\\hline
    \end{tabular} \end{center}
\end{exe}

\end{document}
