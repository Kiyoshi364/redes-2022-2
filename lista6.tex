\documentclass{article}

\usepackage[width=14cm, left=3cm, top=2cm]{geometry}

\usepackage[brazil]{babel}
\usepackage[T1]{fontenc}
\usepackage[utf8]{inputenc}

\usepackage{amsmath}
\usepackage{amssymb}

\usepackage{multicol}

\usepackage{ifthen}
%\usepackage{minted}

\usepackage{tikz}
\usepackage{pgfplots}
\pgfplotsset{compat=newest}
\usepgfplotslibrary{external}
\tikzexternalize

\newcommand{\blank}{\rule[0pt]{5em}{.3pt}}
\newcommand{\nobody}{ninguém}
\newcommand{\preamble}[2]{\noindent%
    Fiz esse trabalho com a ajuda de {\bfseries #1}
    e consultei {\bfseries #2}.
    A versão final do trabalho foi feita
    por mim de forma independente.
    Respostas sem no mínimo 3 frases de justificativa
    não contam ponto.
    \par\noindent Assinatura: \blank\blank\bigskip}

\newcounter{exe-list}
\newenvironment{exe-list}
    {\begin{list}{\alph{exe-list}.}{\usecounter{exe-list}}}
    {\end{list}}

\newenvironment{exe}[2][Problema]
    {\newcommand{\opt}{(Opcional)}%
    \newcommand{\sketch}[1]{{\bfseries Rascunho:} ##1}%
    \medskip\par\noindent\ifthenelse{\equal{#1}{}}
        {\textbf{\large #2}}
        {\textbf{\large #1~#2}}%
    \medskip\par\noindent}
    {\medskip}

\title{Redes de Computadores - Lista 6}
\author{Daniel Kiyoshi Hashimoto Vouzella de Andrade - 119025937}
\date{Dezembro 2022}

\begin{document}
\maketitle

\begin{exe}{8}
    \begin{exe-list}
    \item
        Derivando \(f(p) := N \; p \; (1 - p)^{N-1}\)
        em relação a \(p\):
        \begin{align*}
            \frac{d \; f}{dp} &=
            N \; \left[
                -(N-1) \; p \; (1 - p)^{N-2} + (1 - p)^{N-1}
            \right] \\
            &= N \; ( (1 - p) - (N-1) \; p ) \; (1 - p)^{N-2} \\
            &= N \; ( 1 - p - N \; p + p ) \; (1 - p)^{N-2} \\
            &= N \; ( 1 - N \; p ) \; (1 - p)^{N-2} \\
        \end{align*}
        As raízes são \(p = \frac1N\) e \(p = 1\).
        Então, devemos observar os pontos
        \( \left\{ 0, \frac1N, 1 \right\}\),
        (as raízes e os limites):
        \begin{align*}
            f(0)
                &= N \cdot 0 \; (1 - 0)^{N-1} \\
                &= 0 \\\\
            f\left(\frac1N\right)
                &= N \; \frac1N \; \left(1 - \frac1N\right)^{N-1} \\
                &= \left(1 - \frac1N\right)^{N-1} \\\\
            f(1)
                &= N \cdot 1 \; (1 - 1)^{N-1} \\
                &= 0
        \end{align*}
        O \(p\) que maximiza \(f\) é \(p = \frac1N\).
    \item
        Usando que \(p = \frac1N\):
        \begin{align*}
            f(p)
            &= \left(1 - \frac1N\right)^{N-1} \\
            &= \left(1 - \frac1N\right)^N \; \frac{1}{1 - \frac1N}
        \end{align*}
        Quando \(N \to +\infty\),
        \(\left(1 - \frac1N\right)^N = \frac1e\):
        \begin{align*}
            f(p)
            &= \frac1e \; \frac{1}{1 - \frac1N} \\
            &\approx \frac1e
        \end{align*}
    \end{exe-list}
\end{exe}

\begin{exe}{9}
    Calculando a derivada da eficiência do pure ALOHA
    \(g(p) := N \; p \; (1 - p)^{2(N-1)}\):
    \begin{align*}
        \frac{d \; g}{dp}
        &= N \; \left[
            -2 \; (N-1) \; p \; (1 - p)^{2N-3} + (1 - p)^{2N-2}
        \right] \\
        &= N \; ( -2 \; (N-1) \; p + (1 - p) ) \; (1 - p)^{2N-3} \\
        &= N \; ( -2N \; p + 2 p + 1 - p ) \; (1 - p)^{2N-3} \\
        &= N \; ( -2N \; p + p + 1 ) \; (1 - p)^{2N-3} \\
        &= N \; (1 - (2N-1) \; p) \; (1 - p)^{2N-3} \\
    \end{align*}
    As raízes são \(p = \frac{1}{2N - 1}\) e \(p = 1\).
    Então, devemos observar os pontos
    \( \left\{ 0, \frac{1}{2N-1}, 1 \right\}\),
    (as raízes e os limites):
    \begin{align*}
        g(0)
            &= N \cdot 0 \; (1 - 0)^{2(N-1)} \\
            &= 0 \\\\
        g\left(\frac{1}{2N-1}\right)
            &= N \; \frac{1}{2N-1} \; \left(1 - \frac{1}{2N-1}\right)^{2(N-1)} \\\\
        g(1)
            &= N \cdot 1 \; (1 - 1)^{2(N-1)} \\
            &= 0
    \end{align*}
    O \(p\) que maximiza \(g\) é \(p = \frac{1}{2N-1}\).
    Usando que \(p = \frac{1}{2N-1}\):
    \begin{align*}
        g(p)
        &= N \; \frac{1}{2N-1} \; \left(1 - \frac{1}{2N-1}\right)^{2(N-1)} \\
        &= \frac{N}{2N-1}
            \; \left(1 - \frac{1}{2N-1}\right)^{2N-1}
            \; \frac{1}{1 - \frac{1}{2N-1}} \\
    \end{align*}
    Quando \(N \to +\infty\),
    \(\left(1 - \frac{1}{2N-1}\right)^\frac{1}{2N-1}= \frac1e\) e
    \(\frac{N}{2N-1} = \frac12\):
    \begin{align*}
        g(p)
        &= \frac12 \; \frac1e \; \frac{1}{1 - \frac{1}{2N-1}} \\
        &\approx \frac{1}{2e}
    \end{align*}
\end{exe}

\begin{exe}{10}
    \begin{exe-list}
    \item
        A probabilidade de \(A\) transmitir sem colisão é
        \(s_A := p_A \; (1 - p_B)\),
        similarmente,
        a probabilidade de \(B\) transmitir sem colisão é
        \(s_B := p_B \; (1 - p_A)\).
        O throughput de \(A\) é \(s_A\) pacotes por slot.
        A eficiência total é \(s_A + s_B\).
    \item
        Com \(p_A = 2 \; p_B\):
        \begin{align*}
            s_A &= 2 \; p_B \; (1 - p_B) \\\\
            s_B &= p_B \; (1 - 2 \; p_B) \\
            &= 2 \; p_B \; \left( \frac12 - p_B \right) \\
        \end{align*}
        Restringindo \(s_A = 2 \; s_B\):
        \begin{align*}
            s_A &= 2 \; s_B \\
            2 \; p_B \; (1 - p_B) &= 2 \; p_B \; (1 - 2 \; p_B) \\
            1 - p_B &= 1 - 2 \; p_B \\
            p_B &= 0 \\
        \end{align*}
        Então não dá para achar \(p_A\) e \(p_B\),
        com \(p_A = 2 \; p_B\) e \(p_B \ne 0\),
        tais que \(s_A = 2 \; s_B\).
        Tentando achar sem essa restrição:
        \begin{align*}
            s_A &= 2 \; s_B \\
            p_A \; (1 - p_B) &= 2 \; p_B \; (1 - p_A) \\
            p_A - p_A \; p_B &= 2 \; p_B - 2 \; p_A \; p_B \\
            p_A + p_A \; p_B &= 2 \; p_B \\
        \end{align*}
    \item
        No caso genérico,
        com um total de \(N\) nós,
        a probabilidade de \(A\) transmitir sem colisão é
        \(s_A := 2 \; p \; (1 - p)\),
        similamente,
        a probabilidade de outro nó transmitir sem colisão é
        \(s_O := p \; (1 - 2 \; p)\).
        Dessa forma,
        o throughput de \(A\) é \(s_A\) pacotes por slot e
        o throughput de qualquer outro nó é \(s_O\) pacotes por slot.
    \end{exe-list}
\end{exe}

\begin{exe}{11}
    A probabilidade de algum nó
    realizar uma transmissão com sucesso é
    \(s := p \; (1 - p)^3\).
    \begin{exe-list}
    \item
        Como \(A\) deve ``falhar'' 4 vezes
        e ``dar certo'' na slot 5,
        a probabilidade é \(s \; (1 - s)^4\).
    \item
        Como ``ser no slot 4'',
        não faz diferênça,
        a probabilidade é a soma
        das probabilidades de cada um ``dar certo'',
        ou seja, \(4 \; s\).
    \item
        A probabilidade dos 4 nós ``falharem'' é
        \(4 \; (1 - s)\),
        então isso deve acontecer 2 vezes,
        seguido de um sucesso (o complementar disso),
        ou seja, \(4 \; (1 - s) \; (1 - 4 \; (1 - s))\).
    \item
        A eficiência é igual a probabilidade de ter algum sucesso,
        ou seja, \(4 \; s\).
    \end{exe-list}
\end{exe}

\begin{exe}{12}
    Usando a eficiencia do slotted ALOHA
    \(f_N(p) := N \; p \; (1 - p)^{N-1}\)
    de vermelho,
    e do pure ALOHA
    \(g_N(p) := N \; p \; (1 - p)^{2(N-1)}\)
    de azul.
    \begin{exe-list}
    \item \(N = 15\) \par
        \begin{center}\begin{tikzpicture}
            \begin{axis}[xmin=-0.05,xmax=1.05,ymin=-0.05,ymax=0.4,
                    xlabel=\(p\), ylabel=Eficiência,
                ]
                \addlegendentry{\(f_{15}(p)\)}
                \addplot[domain=0:1, color=red, smooth, thick, samples=100]
                    { 15 * x * (1-x)^(14) };
                \addlegendentry{\(g_{15}(p)\)}
                \addplot[domain=0:1, color=blue, smooth, thick, samples=100]
                 { 15 * x * (1-x)^(29) };
            \end{axis}
        \end{tikzpicture}\end{center}
    \item \(N = 25\) \par
        \begin{center}\begin{tikzpicture}
            \begin{axis}[xmin=-0.05,xmax=1.05,ymin=-0.05,ymax=0.4,
                    xlabel=\(p\), ylabel=Eficiência,
                ]
                \addlegendentry{\(f_{25}(p)\)}
                \addplot[domain=0:1, color=red, smooth, thick, samples=100]
                    { 25 * x * (1-x)^(24) };
                \addlegendentry{\(g_{25}(p)\)}
                \addplot[domain=0:1, color=blue, smooth, thick, samples=100]
                 { 25 * x * (1-x)^(49) };
            \end{axis}
        \end{tikzpicture}\end{center}
    \item \(N = 35\) \par
        \begin{center}\begin{tikzpicture}
            \begin{axis}[xmin=-0.05,xmax=1.05,ymin=-0.05,ymax=0.4,
                    xlabel=\(p\), ylabel=Eficiência,
                ]
                \addlegendentry{\(f_{35}(p)\)}
                \addplot[domain=0:1, color=red, smooth, thick, samples=100]
                    { 35 * x * (1-x)^(34) };
                \addlegendentry{\(g_{35}(p)\)}
                \addplot[domain=0:1, color=blue, smooth, thick, samples=100]
                 { 35 * x * (1-x)^(59) };
            \end{axis}
        \end{tikzpicture}\end{center}
    \end{exe-list}
\end{exe}

\begin{exe}{15}
    \begin{exe-list}
    \item
        Não, Host E tem um ``caminho direto'' para Host F e
        estão na mesma rede.
        Source IP: IP-E, Source MAC: MAC-E,
        Target IP: IP-F, Target MAC: MAC-F.
    \item
        Não, Host B não está na mesma rede que Host E,
        vai mandar o datagram para o Router R1.
        Source IP: IP-E, Source MAC: MAC-E,
        Target IP: IP-B, Target MAC: MAC-R1.
    \item
        Switch S1 vai repassar o request de ARP para
        todos as outras máquinas (B, C, D, R1),
        pois é um broadcast.
        Router R1 vai receber o request,
        mas não vai repassar para a rede 3.
        Quando Host B receber o request,
        ele já sabe o MAC do Host A,
        pois ele está escrito no request.
        O switch S1 vai repassar apenas para o Host A,
        o request ARP tinha o MAC do Host A,
        então o switch S1 já ``aprendeu''
        qual interface tem que usar para chegar ao Host A.
    \end{exe-list}
\end{exe}

\begin{exe}{20}
    \begin{exe-list}
    \item
        A probabilidade de um nó começar uma transmissão,
        dado que o sistema está no estado improdutivo, é
        \(s = p \; (1-p)^{N-1}\).
        Em um ciclo, vamos ter um período produtivo de \(k\) slots
        e um improdutivo de, na média,
        esperança do tempo que demora para ocorrer
        o primeiro início de transmissão,
        ou seja, a esperança da geométrica de \(N \; s\).
        Assim, o tamanho do ciclo é
        \(k + \frac{1}{N \; s}\) e
        a eficiência é
        \[
            \frac{k}{k + \frac{1}{N \; s}}
            = \frac{k}{k + \frac{1}{N \; p \; (1-p)^{N-1}}}
        \]
    \item
        Simplificando \(h_N(p)\):
        \begin{align*}
            h_N(p)
            &= \frac{k}{k + \frac{1}{N \; p \; (1-p)^{N-1}}} \\
            &= \frac{k}{\frac{k \; N \; p \; (1-p)^{N-1} + 1}{N \; p \; (1-p)^{N-1}}} \\
            &= \frac{k \; N \; p \; (1-p)^{N-1}}{k \; N \; p \; (1-p)^{N-1} + 1} \\
        \end{align*}
        Derivando \(h_N(p)\) em relação a \(p\):
        \begin{align*}
            \frac{d \; h_N(p)}{dp}
            &= \frac{d \;}{dp} \left( \frac{k \; N \; p \; (1-p)^{N-1}}{k \; N \; p \; (1-p)^{N-1} + 1} \right) \\
            &= \frac{d \;}{dp} \left( \frac{f(p)}{f(p) + 1} \right) \\
            &= \frac{d \;}{dp} \left( f(p) \; \frac{1}{f(p) + 1} \right) \\
            &= f(p) \; \frac{d \;}{dp} \left( \frac{1}{f(p) + 1} \right)
                + \frac{d \; f(p)}{dp} \; \frac{1}{f(p) + 1} \\
            &= f(p) \; \left( \frac{-1}{(f(p) + 1)^2} \; f'(p) \right)
                + f'(p) \; \frac{1}{f(p) + 1} \\
            &= \frac{- f(p) \; f'(p)}{(f(p) + 1)^2} + \frac{f'(p)}{f(p) + 1} \\
            &= \frac{(f(p) + 1) \; f'(p) - f(p) \; f'(p)}{(f(p) + 1)^2} \\
            &= \frac{f(p) \; f'(p) + f'(p) - f(p) \; f'(p)}{(f(p) + 1)^2} \\
            &= \frac{f'(p)}{(f(p) + 1)^2} \\
            &= \frac{k \; N \; (1-p)^{N-1} - k \; N \; (N-1) \; p \; (1-p)^{N-2}}{(k \; N \; p \; (1-p)^{N-1} + 1)^2} \\
            &= \frac{k \; N \; (1-p)^{N-2} ((1-p) - (N-1) \; p)}{(k \; N \; p \; (1-p)^{N-1} + 1)^2} \\
            &= \frac{k \; N \; (1-p)^{N-2} (1 - p - N \; p + p)}{(k \; N \; p \; (1-p)^{N-1} + 1)^2} \\
            &= \frac{k \; N \; (1-p)^{N-2} (1 - N \; p)}{(k \; N \; p \; (1-p)^{N-1} + 1)^2} \\
        \end{align*}
        As raízes são \(\left\{ \frac1N, 1 \right\}\).
        \begin{align*}
            h_N(0)
                &= \frac{k \; N \cdot 0 \; (1-0)^{N-1}}{k \; N \cdot 0 \cdot (1-0)^{N-1} + 1} \\\\
                &= \frac{0}{1} \\
                &= 0 \\\\
            h_N\left(\frac1N\right)
                &= \frac{k \; N \; \frac1N \; (1-\frac1N)^{N-1}}{k \; N \; \frac1N \; (1-\frac1N)^{N-1} + 1} \\
                &= \frac{k \; (1-\frac1N)^{N-1}}{k \; (1-\frac1N)^{N-1} + 1} \\\\
            h_N(1)
                &= \frac{k \; N \cdot 1 \cdot (1-1)^{N-1}}{k \; N \cdot 1 \cdot (1-1)^{N-1} + 1} \\
                &= \frac{0}{1} \\
                &= 0
        \end{align*}
        Então, \(p^* = \frac1N\).
    \item
        Foi respondida no item anterior,
        com \(p^* = \frac1N\):
        \[
            h_N(p^*) =
            \frac{k \; (1-\frac1N)^{N-1}}{k \; (1-\frac1N)^{N-1} + 1}
        \]
    \item
        Para que a razão se aproxime a \(1\),
        \(k \; (\frac{N-1}{N})^{N-1}\) tem que muito grande.
        Com \(N\) muito grande, isso se aproxima a
        \(k \; \frac1e\),
        que não é ``tão grande''.
        Mas com \(k\) muito grande, isso se aproxima
        a um valor muito grande.
        Com \(k\) muito grande,
        o tamanho do frame \(L = k \; N \; S\)
        também fica muito grande.
    \end{exe-list}
\end{exe}

\end{document}
