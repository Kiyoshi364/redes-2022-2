\documentclass{article}

\usepackage[width=14cm, left=3cm, top=2cm]{geometry}

\usepackage[brazil]{babel}
\usepackage[T1]{fontenc}
\usepackage[utf8]{inputenc}

\usepackage{amsmath}
\usepackage{amssymb}

\usepackage{multicol}

\usepackage{ifthen}
%\usepackage{minted}

\newcommand{\blank}{\rule[0pt]{5em}{.3pt}}
\newcommand{\nobody}{ninguém}
\newcommand{\preamble}[2]{\noindent%
    Fiz esse trabalho com a ajuda de {\bfseries #1}
    e consultei {\bfseries #2}.
    A versão final do trabalho foi feita
    por mim de forma independente.
    \par\noindent Assinatura: \blank\blank\bigskip}

\newcounter{exe-list}
\newenvironment{exe-list}
    {\begin{list}{\alph{exe-list}.}{\usecounter{exe-list}}}
    {\end{list}}

\newenvironment{exe}[2][Problema]
    {\newcommand{\opt}{(Opcional)}%
    \newcommand{\sketch}[1]{{\bfseries Rascunho:} ##1}%
    \medskip\par\noindent\ifthenelse{\equal{#1}{}}
        {\textbf{\large #2}}
        {\textbf{\large #1~#2}}%
    \medskip\par\noindent}
    {\medskip}

\newcommand{\Declare}[3]{\DeclareMathOperator{#1}{#2{}#3}}
\newcommand{\DeclareBit}[2]{\Declare{#1}{#2}{b}}
\DeclareBit{\bit}{}
\DeclareBit{\kbit}{k}
\DeclareBit{\Mbit}{M}
\DeclareBit{\Gbit}{G}
\newcommand{\DeclareByte}[2]{\Declare{#1}{#2}{byte}}
\DeclareByte{\byte}{}
\DeclareByte{\kbyte}{k}
\DeclareByte{\Mbyte}{M}
\DeclareByte{\Gbyte}{G}
\newcommand{\DeclareBps}[2]{\Declare{#1}{#2}{bps}}
\DeclareBps{\bps}{}
\DeclareBps{\kbps}{k}
\DeclareBps{\Mbps}{M}
\DeclareBps{\Gbps}{G}

\title{Redes de Computadores - Lista 2}
\author{Daniel Kiyoshi Hashimoto Vouzella de Andrade - 119025937}
\date{Setembro 2022}

\begin{document}
\maketitle

\preamble{\nobody}{\nobody}

\begin{exe}[]{Pergunta 1}
    \begin{itemize}
        \item \textbf{Non-persistent HTTP:}
            cada requisição precisa de uma conexão,
            precisa de várias conexões para receber vários objetos.
        \item \textbf{Persistent HTTP:}
            em uma conexão pode ser feitos várias requisições,
            isso remove overhead de criar uma conexão
            quando se requer vários objetos.
    \end{itemize}
\end{exe}

\begin{exe}[]{Pergunta 2}
    \begin{itemize}
        \item Os cookies servem para o servidor ``lembrar''
            com quem está conversando.
        \item HTTP é stateless. TCP é statefull.
        \item Isso não está relacionado.
            TCP ``simplesmente'' transporta bits de
            um Host para outro Host,
            e acaba usando estado para decidir
            quando enviar mais ou menos dados.

            O HTTP descreve um par requisição e resposta,
            e esse par (a princípio) não tem nenhuma informação sobre
            ``pares passados'', então é dito stateless.
    \end{itemize}
\end{exe}

\begin{exe}[]{Pergunta 3}
    Usa UDP,
    pois tem que responder muitas mensagens
    (todo mundo quer saber como chegar no www.sitelegal.com)
    e perda de dados, ou a não garantia da resposta,
    não é tão ruim, basta refazer o request.
    (Na primeira vez, a busca pelo IP pode ser custoso,
    mas nas próximas a resposta vai estar no cache,
    então é acaba ficando de ``graça'')
\end{exe}

\begin{exe}[]{Pergunta 4}
    A requesição recursiva é muito conveniente,
    só precisa fazer uma requisição e
    o servidor requisitado vai resolver o resto;
    mas congestiona o servidor em questão,
    pois todo mundo iria passar por esse servidor
    para saber a resposta.

    A requesição iterativa faz o cliente fazer as próximas buscas,
    isso permite que o servidor fique menos ocupado
    e ajuda a garantir que estará servindo por mais tempo.
\end{exe}

\begin{exe}[]{Pergunta 5}
    Vantagens:
    \begin{itemize}
        \item Mais resistente:
            Se o server cair, tudo cai.
            Se um nó num p2p cai, o resto funciona.
        \item Mais barato:
            Manter um servidor ligado o tempo todo é caro.
        \item Cresce com facilidade:
            Qualquer um pode se comportar como ``servidor''.
        \item Potencialmente mais rápida:
            não precisa de intermediário (servidor) ou
            pode se conseguir fragmentos de informação
            por vários nós diferentes ao mesmo tempo.
    \end{itemize}
    Desvantagens:
    \begin{itemize}
        \item Segurança:
            qualquer um pode ser um ``servidor'',
            é difícil garantimos que um dado nó é confiável/seguro.
        \item Inconsistência:
            atualizar uma informação demora tempo,
            já que a informação vai ter que passar por todos
            os pontos.
    \end{itemize}
\end{exe}

\end{document}
