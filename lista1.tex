\documentclass{article}

\usepackage[width=14cm, left=3cm, top=3cm]{geometry}

\usepackage[brazil]{babel}
\usepackage[T1]{fontenc}
\usepackage[utf8]{inputenc}

\usepackage{amsmath}
\usepackage{amssymb}

\usepackage{multicol}

\usepackage{minted}

\newcommand{\blank}{\rule[0pt]{5em}{.3pt}}
\newcommand{\nobody}{ninguém}
\newcommand{\preamble}[2]{\noindent%
    Fiz esse trabalho com a ajuda de {\bfseries #1}
    e consultei {\bfseries #2}.
    A versão final do trabalho foi feita
    por mim de forma independente.
    \par\noindent Assinatura: \blank\blank\bigskip}

\newcounter{exe-list}
\newenvironment{exe-list}
    {\begin{list}{\alph{exe-list}.}{\usecounter{exe-list}}}
    {\end{list}}

\newenvironment{exe}[2][]
    {\newcommand{\opt}{(Opcional)}%
    \newcommand{\sketch}[1]{{\bfseries Rascunho:} ##1}%
    \newcommand{\nop}{\hspace{-1ex}}%
    \medskip\par\noindent\ifthenelse{\equal{#1}{}}
        {\textbf{\large Problema~#2}}
        {\textbf{\large #1~#2}}%
    \medskip\par\noindent}
    {\medskip}

\newcommand{\DeclareBps}[2]{\DeclareMathOperator{#1}{#2{}bps}}
\DeclareBps{\bps}{}
\DeclareBps{\kbps}{k}
\DeclareBps{\Mbps}{M}
\DeclareBps{\Gbps}{G}

\title{Redes de Computadores - Lista 1}
\author{Daniel Kiyoshi Hashimoto Vouzella de Andrade - 119025937}
\date{Setembro 2022}

\begin{document}
\maketitle

\preamble{\nobody}{\nobody}

\begin{exe}{3 \opt}
    \begin{exe-list}
    \item \sketch{Circuit-switched,
        minha intuição diz que
        \emph{reservar} uma \emph{linha} é muito caro,
        mas como a \emph{linha} vai ser usada por muito tempo
        acaba valendo mais a pena do que
        pagar muitas vezes o pequeno custo dos pacotes.}
    \item Não, com a premissa de que todos os nós são
        poderosos o suficiente para aguentar todas as aplicações
        enviando dados ao mesmo tempo
        (isso foi o que eu entendi do enunciado),
        não vai ter nenhuma perda e o atraso será mínimo.
    \end{exe-list}
\end{exe}

\begin{exe}{4 \opt}
    \begin{exe-list}
    \item 8.
        \sketch{O \(A\) pode ter no máximo 4 conexões,
        para ``economizar'' links,
        vamos supor que as 4 são para \(B\).
        O \(C\) também pode ter no máximo 4 conexões,
        agora as 4 apenas para \(D\).
        Falta mostrar que não tem uma configuração melhor.}
    \item 8, pois são 4 passando pelo \(B\) e 4 passando pelo \(D\).
    \item Sim, basta \(A\) fazer 2 conexões passando por \(B\) e
        2 passando por \(D\) e, analogamente,
        \(B\) fazer 2 conexões passando por \(A\) e
        2 passando por \(C\).
    \item \sketch{Não, foi discutido em aula.
        O ``caminho mais curto'' pode estar
        \emph{``congestionado''}, então é melhor usar um
        mais longo e ``livre''.}
    \end{exe-list}
\end{exe}

\begin{exe}{5 \opt}
    \begin{exe-list}
    \item \begin{itemize}
        \item Tempo para transmitir todos os \(10\) carros:
            \( \frac{10}{5} = 2 \text{ min} \)
        \item Tempo para o último carro chegar no próximo pedágio:
            \( \frac{150}{100} = 1.5 \text{ h} \)
        \item Tempo total:
            \( 2 \cdot 1.5 \text{ h} + 2 \cdot 2 \text{ min}
                = 3 \text{ h } 4 \text{ min } \)
    \end{itemize}
    \item \begin{itemize}
        \item Tempo para transmitir todos os \(8\) carros:
            \( \frac{8}{5} = 1.6 \text{ min} \)
        \item Tempo para o último carro chegar no próximo pedágio
            (igual): \( 1.5 \text{ h} \)
        \item Tempo total:
            \( 2 \cdot 1.5 \text{ h} + 2 \cdot 1.6 \text{ min}
                = 3 \text{ h } 3 \text{ min } 12 \text{ s } \)
    \end{itemize}
    \item \begin{itemize}
        \item Tempo para transmitir o único carro:
            \( \frac{1}{5} = 12 \text{ s} \)
        \item Tempo para o único carro chegar no próximo pedágio
            (igual): \( 1.5 \text{ h} \)
        \item Tempo total:
            \( 2 \cdot 1.5 \text{ h} + 2 \cdot 12 \text{ s}
                = 3 \text{ h } 24 \text{ s } \)
    \end{itemize}
    \end{exe-list}
\end{exe}

\newpage
\begin{exe}{6}
    \begin{multicols}{2}
    \begin{exe-list}
    \item \[ d_{prop} = \frac{m}{s} \]
    \item \[ d_{trans} = \frac{L}{R} \]
    \item \[ d = d_{trans} + d_{prop} \]
    \item Está a caminho do Host B (acabou de sair do Host A).
    \item Está a caminho do Host B:
        falta \(d_{prop} - d_{trans}\) segundos para chegar no Host B,
        logo está a \((d_{prop} - d_{trans}) \; s\) metros do Host B.
    \item Já chegou no Host B.
    \item \begin{align*}
            d_{prop} &= d_{trans} \\
            \frac{m}{s} &= \frac{L}{R} \\
            \frac{m}{2.5 \cdot 10^8} &= \frac{120}{56} \\
            m &= \frac{120}{56 \cdot 2.5 \cdot 10^8} \\
            m &= \frac{3}{35} \; 10^{-7} \approx 8.57 \cdot 10^{-9}
        \end{align*}
    \end{exe-list}
    \end{multicols}
\end{exe}

\begin{exe}{7 \opt}
    \begin{itemize}
        \item Pacotes enviados por segundo:
            \( \frac{56 * 8}{64} = 7 \)
        \item Tempo da ``criação'' do primeiro bit
            até o pacote ficar pronto, em segundos:
            \( \frac{1}{7} = 0.142857 \)
        \item Tempo de transmissão, em segundos:
            \( \frac{56 * 8}{2 \Mbps} = 0.26 \)
        \item Tempo de propagação, em segundos:
            \( 0.010 \)
        \item Tempo total, em segundos:
            \( 0.142857 + 0.26 + 0.010 = 0.412857 \)
    \end{itemize}
\end{exe}

\begin{exe}{8}
    \begin{exe-list}
    \item \[ \frac{3 \Mbps}{150 \kbps} = 20 \]
    \item Lendo o enunciado: \( 0.1 \)
    \item \( N \sim Binom(120, 0.1) \):
        \[ P(N = n) = \binom{120}{n} \; (0.1)^n \; (0.9)^{120-n} \]
    \item \(0.00794119\)
        \par Código em \texttt{J} (apenas para registro):
        \begin{minted}{J}
    pbinom =: 2 : '((!&u) * (v&^) * (1-v) ^ (u&-))'
    1 - +/ (120 pbinom 0.1) i. 21
0.00794119
        \end{minted}
    \end{exe-list}
\end{exe}

\begin{exe}{9}
    \begin{exe-list}
    \item \[ N = \frac{1 \Gbps}{100 \kbps} = 10000 \]
    \item Para evitar confusão com \(N\) do último item,
        eu uso \(n\) no seu lugar;
        e \(A\) como v.a. que representa
        o número de usuários ativos.
        \[ P(A > n)
            = 1 - P(A \le n)
            = 1 - \sum_{i = 0}^n \binom{M}{i} \; p^i \; (1-p)^{M-i} \]
    \end{exe-list}
\end{exe}

\begin{exe}{10}
    \begin{itemize}
        \item Delay de transmissão \(i \in \{ 1, 2, 3 \}\):
            \( \frac{L}{R_i} \)
        \item Delay de propagação \(i \in \{ 1, 2, 3 \}\):
            \( \frac{d_i}{s_i} \)
        \item Delay de processamento \(i \in \{ 1, 2, 3 \}\):
            \( d_{proc} \)
        \item Delay total:
            \[ 3 \; d_{proc}
                + \sum_{i = 1}^3 \frac{L}{R_i} + \frac{d_i}{s_i}
            \]
        \item \(L := 1500 \cdot 2^8\), \(s_i = 2.5 \cdot 10^8\),
            \(R_i := 2 \Mbps\), \(d_{proc} := 0.003\),
            \(d_1 = 5 \cdot 10^6\), \(d_2 = 4 \cdot 10^6\),
            \(d_3 = 1 \cdot 10^6\):
            \begin{align*}
                3 \cdot 0.003
                    + \frac{1500 \cdot 2^8}{2 \cdot 10^6}
                    + \frac{1}{2.5 \cdot 10^8} \sum_{i = 1}^3 d_i \\
                0.009 + \frac{1500 \cdot 2^7}{10^6}
                    + \frac{10^7}{2.5 \cdot 10^8} \\
                0.009 + 0.192 + \frac{1}{25} \\
                0.009 + 0.192 + 0.040 = 0.281 \\
            \end{align*}
    \end{itemize}
\end{exe}

\begin{exe}{16}
    \[
        a = \frac{N}{d} = \frac{10 + 1}{0.010} = 1100
    \]
\end{exe}

\begin{exe}[\nop]{Vídeo}
    O vídeo deu uma explicação intuitiva de como a
    Lei de Little funciona e porque faz sentido ser verdade.
    Deu uma motivação para um possível uso da Lei,
    mostrou um gráfico e duas formas diferentes
    de calcular sua área e a partir dessas duas formas
    enunciou a Lei.

    Perguntas:
    \begin{itemize}
        \item Como é a prova formal da Lei de Little?
        \item Como foram feitas as animações dos gráficos e fórmulas?
    \end{itemize}
\end{exe}

\end{document}
